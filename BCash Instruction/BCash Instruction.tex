\documentclass[10pt, a4paper]{article}

\usepackage{polyglossia}
\usepackage[left=2.5cm, right=2cm, top=2.5cm, bottom=2cm]{geometry}

\setdefaultlanguage[numerals=Devanagari]{bengali}
\newfontfamily\bengalifont[Script=Bengali]{Kalpurush}

\title{bcash Instruction}

\begin{document}

\section*{Send Money, সেন্ড মানি}
একজনের পারসোনাল বিক্যাশ নম্বর থেকে আরেকজনের পারসোনাল বিক্যাশ নম্বরে টাকা পাঠাতে হলে সেন্ড মানি করতে হবে। সেন্ড মানি করার ধাপসমূহ: \\
\begin{enumerate}
	\item[১.] যেই নম্বরে বিক্যাশ খোলা আছে সেই নম্বরে *247\#  ডায়াল করতে হবে।
	\item[২.] তারপর 1 লিখে রিপলাই(Replay) পাঠাতে হবে।
	\item[৩.] তারপর যার বিক্যাশ নম্বরে টাকা পাঠাতে হবে তার বিক্যাশ নম্বরটি লিখে রিপলাই(Replay) পাঠাতে হবে।
	\item[৪.] তারপর টাকার পরিমাণ ইংরেজিতে লিখে রিপলাই(Replay) পাঠাতে হবে। টাকা ১ হাজার পাঠাতে চাইলে 1000, ২ হাজার পাঠাতে চাইলে 2000। এইভাবে ইংরেজিতে লিখে রিপলাই(Replay) পাঠাতে হবে।
	\item[৫.] তারপর আবার 1 লিখে রিপলাই(Replay) পাঠাতে হবে।
	\item[৬.] তারপর নিজের বিক্যাশ PIN NUMBER লিখে রিপলাই(Replay) পাঠাতে হবে।
\end{enumerate}

\section*{Cash Out, ক্যাশ আউট}

\end{document}