\documentclass[11 pt]{article}

\usepackage[margin=2cm]{geometry}
\usepackage{gensymb} % degree symbol

\title{Fourier Analysis}
\date{13 August, 2021 \\ 10:58 PM}

\begin{document}

\maketitle

\section*{Prerequisites}
\begin{enumerate}
	\item Periodic Function: $\sin x$ and $\cos x$ having period $2\pi$, $\tan x$ having period only $\pi$
	\item Trigonometry
	\item Basic Integration
\end{enumerate}

\section*{Basic}
We can write any function as a series and can modify it with \textbf{Integration}. Joseph Fourier$(1768 \leftarrow 1830)$ derived a special trigonometric series.
Trigonometric series is a series of the form:
$$\frac{A_0}{2} + \sum_{n=1}^{\infty} (A_n\cos nx + B_n\sin nx)$$
It will called \textbf{Fourier Series} if the terms $A_0, A_n, B_n$ is:
$$A_0 = \frac{1}{2\pi}\int_{-\pi}^{\pi} f(x)\cdot dx$$
$$A_n = \frac{1}{\pi}\int_{-\pi}^{\pi} f(x)\cdot \cos nx\cdot dx$$
$$B_n = \frac{1}{\pi}\int_{-\pi}^{\pi} f(x)\cdot \sin nx\cdot dx$$
Where \textbf{$f(x)$} is any single-valued function defined in interval $(-\pi, \pi)$.

\section*{Some helpful equations}
\begin{enumerate}
	\item $\sin 0\degree = \sin \pi = 0$
	\item $\cos 0\degree = \cos 2n\pi = (-1)^{2n} = 1$
	\item $\cos n\pi = (-1)^n$
	\item $\frac{d}{dx} \sin\theta = \cos\theta$
	\item $\frac{d}{dx} \cos\theta = -\sin\theta$
	\item $\int \sin\theta = -\cos\theta$
	\item $\int \cos\theta = \sin\theta$
\end{enumerate}

\end{document}