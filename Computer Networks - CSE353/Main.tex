\documentclass{article}

% preamble of packages
\usepackage[a4paper]{geometry}
\usepackage{hyperref}

\author{Sofiullah Iqbal Kiron}
\title{CSE353: Computer Networks \linebreak \& \linebreak CSE354: Computer Networks Lab}

\begin{document}

\maketitle

\section*{Course Credentials}
\begin{flushright}
    Course Teacher: Md. Monoyar Hossain, Assistant Professor, \linebreak Dept. of CSE. \linebreak \textbf{BSMRSTU}.
    \linebreak
    Sunday, Wednesday at 10:30 AM.
    \linebreak
    Reference book: vule geci.
\end{flushright}

\section*{Main topics}
\begin{enumerate}
    \item Network Topology
    \item Mobile network
    \item Computer network
\end{enumerate}

\section*{Some questions answered in the class}

\begin{enumerate}
    \item What is IP, MAC, DNS?
    \item What is the difference between IPV4(32 bits or 4 bytes) and IPV6(128 bits)?
    \item Distinguish between public and private IP address.
    \item What is internet and intranet?
\end{enumerate}

\section*{Mid Topics}
\begin{enumerate}
    \item Basic networking
    \item IP addresses
    \item Subnet
\end{enumerate}

Internet Protocol: is a set of rules for routing and addressing data packets so that they can traverse across networks and arrive at right destination.
In other words, formatting and processing data.
Use of protocols enable computers communicate over the internet regardless.


\section*{Helpful links}
\begin{itemize}
    \item https://www.cloudflare.com/learning/dns/glossary/what-is-my-ip-address/
    \item https://www.freecodecamp.org/news/subnet-mask-definition/
    \item https://support.huawei.com/enterprise/en/doc/EDOC1100145159
\end{itemize}


\section*{Softwares required for lab work}

\subsection*{\href{https://www.wireshark.org/}{WireShark}}
App1

\subsection*{Cisco Packet Tracer}
App2

\end{document}