\documentclass[11 pt]{article}


\usepackage[table, x11names]{xcolor}
\usepackage[document]{ragged2e}
\usepackage[T1]{fontenc}
\usepackage[left=1.8cm, right=1.8cm, top=2cm, bottom=3cm]{geometry}

\usepackage{multicol}

\usepackage
{
 array,
 pgfplots,
 tabularx,
 longtable,
}

% For scripting in Bangla.
\usepackage{polyglossia}
\setdefaultlanguage[numerals=Devanagari]{bengali}
\newfontfamily\bengalifont[Script=Bengali]{Kalpurush}


\begin{document}

\thispagestyle{empty}

	\begin{center}
	\rowcolors{3}{green!35}{green!70}
	\arrayrulecolor{white}
	\columnseprule = 0.5 mm
		\begin{longtable}{|| m{7 em} || m {7 em} || m {7 em} || m{9 em} ||}
			\hline\hline
			\rowcolor{teal!20}
			\multicolumn{4}{c}{\textbf{\textcolor{black}{Number of Calendars to be purchased by Upozilla}}}\\
			\hline\hline
			\rowcolor{red!10}
			\textbf{উপজেলা} & \textbf{সভাপতি} & \textbf{যোগাযোগ} & \textbf{গৃহীত ক্যালেন্ডার সংখ্যা} \\
			মুকসুদপুর & আল-আমিন & 01790305534 & ১০০ \\
			কাশিয়ানি & মো. সাদ্দাম & 01748173757 & ১০০ \\
			গোপালগঞ্জ সদর & আবদুল্লাহ & 01312127645 & ০০০ \\
			কোটালিপাড়া & ইমরান মোল্লা & 01966565945 & ২০০ \\
			টুঙ্গিপাড়া & ইব্রাহিম শেখ & 01961638237 & ১০০ \\
			\hline\hline
		\end{longtable}
	\end{center}

\end{document}