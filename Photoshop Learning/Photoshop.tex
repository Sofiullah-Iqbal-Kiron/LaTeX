\documentclass[11 pt]{article}

\title{What I have learned During Photoshop Course}
\author{Sofiullah Iqbal Kiron}

\begin{document}

\maketitle
\tableofcontents

\begin{enumerate}
	\item alt + tab: for preview all currently opened apps in the middle of screen.
	\item If no files opened, photoshop home window will be views with recent photo related documents.
	\item Photoshop document extension is "PSD". Means - Photoshop document.
	\item Create new PSD by clicking "Create New" at the Home window.
	\item In the main working space, there are, Menubar on top most level of the window.
	\item Photo preview area at the left middle.
	\item Vertical aligned toolbar at left most side.
	\item Options bar for currently selected tool just below of Menubar.
	\item So many control panels on right side.
	\item Preferred four photoshop document type: phg, JPG, tif and PSD.
	\item Can change color of preview canvas background color by right click and selection.
	\item Changing color theme preview icons by pressing: ctrl + alt + shift + click
	\item There is a useful color picker on panels bar of any standards like, RGB, HSB and so forth\dots
	\item Rich tool tip are enabled. Can remove this from Edit $\rightarrow$ Preferences $\rightarrow$ Tools
	\item Can customize toolbar from Edit $\rightarrow$ Toolbar
	\item Object selection tool (W) is more useful. So pretty.
	\item Must be able to working with layers and masks.
	\item Photoshop's ability to work with layers is definitely one of favorite features. Let's dive into to see how we can work with multiple images in a single document. Layers will be shown in panel section of photoshop interface. Or we can hide/show this layer panel from window $\rightarrow$ Layers or directly pressing  F7. Now, we can drag layers from one file and then drag this to another.
	\item ctrl + r: In order to show rulers. We can change rulers unit(pixel, percent etc) by right click above on ruler and then drag ruler guides from rulers.
	\item Photoshop's most powerful image creation tool, layer. Every photo's begins life is flat, no-layers image file. In photoshop, this called background. We can float one image upon another and those all are several. Edit on one image don't affect to other. But we can blend layers together. A document that contains layers is said to be layered composition. Welcome to the power of layers. Layers are stored as stack. It is an example of stack. "alt" button often reverses the natural behavior of an icon/action.
	\item[Canvas:] The physical parameter of the image.
	\item[Windows:] enter key.
	\item[Mac:] return key.
	\item[Magic Wand Tool:] Select same colored pixels.
	\item Multiply blending mode will remove bright pixels. It just keep the dark staff.
	\item[HSB:] Hue, Saturation, Brightness.
	\item[Brush Essentials:] 'b' stands for brush tool. Brush means brush and we can draw by brush on a selected layer. Set the brush property form option bar. To reset brush, just right click and press reset. We can change brush size, hardness, shape and angle also. Can search brush or select a brush preset. Click on folder brush icon in order to bring up brush property panel that hold brush presets, size, shape, hardness and so forth(A number of additional options). Press or hold right square bracket to increase brush size or left square bracket to decrease the size. Another way to increase/decrease brush size, alt + mouse right button press and hold and then right-left "to and fro", up-down "to and fro" for hardness. Press and hold shift key to order the brush paint only straight line, no matter where did you drag your brush. Check all options from options bar for the brush tool. Or we can change opacity of brush color by directly pressing 0-9 keys. 0 to 100\%, 1 to 10\% etc.
	\item Crtl + Backspace to fill the layer with background color. Flow will increase/decrease of painting flow/speed.
	\item Photo Retouching is a art to making something look better. Ctrl + Alt + J to copy currently selected layer and make a new layer, also bring up the rename dialogue box. Can use arrow keys to move up/down/left/right the keys.
	\item 'X' to call swap(foreground\_color, background\_color)
	\item Ctrl + h to hide/show the selection outline.
	\item[Spot Healing Brush:] Allows us to paint with content aware fill. Also we can invoke fill by pressing Shift + Backspace(windows)/delete(mac). Shortcut of healing brush tool: J. It's one kind of surgery, g $\rightarrow$ j. Pretty good.
	\item ctrl + tab: preview next image.
	\item shift + F1/F2 to go to darker/brighter interface. There are two darker and two brighter interface. Preview pasteboard.
	\item ctrl + 0: Fit to page.
\end{enumerate}

% Setting counter of section manually.
\setcounter{section}{10}
\section{Content Aware}
This is the intermediate section. We can design magazine poster in photoshop. Photoshop uses artificial intelligence for content aware. 

\end{document}
