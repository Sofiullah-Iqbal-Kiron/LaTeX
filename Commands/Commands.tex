% Set margin to 2.3cm for mobile view compatibility
% I use "extsizes" package to increase article size manually. This package allows 8pt, 9pt, 10pt, 11pt, 12pt,  14pt, 17pt and 20 pt text. Documentclass must be extarticle. But why the geometry package is not working after including extsizes?
% Syntax text color: \textcolor{red}{}
% Example text color: \textcolor{red!70}{}
% New command for red-teletype text: \R{\T{}}
% F1 (fn + f1 on laptop) for quick build.

\documentclass[12 pt, letterpaper]{extarticle}

\usepackage[document]{ragged2e}
\usepackage[margin=2.3cm]{geometry}
\usepackage
{
	array, % Table formatting.
	longtable, % Long table which takes up two or more pages.
	colortbl,
	graphicx, % \includegraphics and graphics related
	extsizes, % increase extarticle size manually
	hyperref,
	forest
}

\newcommand{\R}{\textcolor{red}} % Just replacement.
\newcommand{\T}{\texttt}

\title{\textcolor{blue}{Bash command line and Git}}
\author
{
	\includegraphics[scale=0.2]{User Profile.jpg} \\ % scale is better scaling for picture. Scale 0.2 means 20 percent will be the printed size of main size.
	\textit{Sofiullah Iqbal Kiron} \\
	\R{\rule{11 cm}{2 pt}}
}
\date{31 March, 2021}

\begin{document}

\maketitle
\justify

If git is already installed in your system, then bash command line interface are installed also. Type and run command from it, the command might be Linux, UNIX or standard git command. All of the git command starts with keyword \R{\T{git}}.\\
Simple bash commands:

\section*{Bash Commands}
\begin{enumerate}
	\item To get help or print manual list of any command just type \R{\T{command --help}}
	\item Sometimes we want to create a new file or there may be times when the requirement is to change the timestamps of a file. The \R{\T{touch [option]...FILE...}} command is primarily used to change file timestamps, but if the file(whose name is passed as an argument with extension) doesn't exist, the the command will create it. Ex. \textcolor{red!70}{\texttt{touch newText.txt}} will create a new text file at working directory by that specified name if not already exist.
	\item To remove a directory and all its contents, including sub-directories and files (cmd command): \R{\T{rm -r "directory name"}} r stands for recursive. That means recursively remove all.
	\item Powershell version: Run as administrator, type \R{\T{\$PSVersionTable}}, hit enter.
	\item Linux/UNIX command to write something on a file: \R{\T{echo message > fileNameWithExtension}} Ex: \textcolor{red!70}{\texttt{echo Hello > file.txt}}
	
	\item Directory:
		\begin{enumerate}
			\item Make: \R{\T{mkdir dirName}}
			\item Change working directory: \R{\T{cd dirNameWithFullPath}}
			\item Path of working directory: \R{\T{pwd}}
		\end{enumerate}
	
	\item \R{\T{Start}}:
		\begin{enumerate}
			\item \R{\T{explorer}}
			\item \R{\T{texmaker}}
			\item \R{\T{msedge}} or \R{\T{msedge <link>}}
			\item \R{\T{chrome}} or \R{\T{chrome <link>}}
			\item \R{\T{brackets}}
			\item \R{\T{opera}} or \R{\T{opera <link>}}
			
			\item \R{\T{cmd}}
			\item \R{\T{powershell}}
		\end{enumerate}
	
	\item List:
		\begin{enumerate}
			\item Simple: \R{\T{ls / dir}}
			\item Hidden included: \R{\T{ls/dir -a}}
			\item Hidden included with details: \R{\T{ls/dir -all}}
		\end{enumerate}
		
	\item Let's some fun:
		\begin{enumerate}
			\item \R{\T{factor n}} will print prime factors of \R{\T{n}}. It is a build-in program like we say.
		\end{enumerate}
	
	\item Open current directory on VS code: \R{\T{code .}}
	\item Stop execution of a running command: \R{\T{ctrl + c}}
	\item If we wanna see all the commands(that we typed and hit enter, not matter right or wrong the command was) as a list: \R{\T{history}}
	\item Clear the history: \R{\T{history -c}}
	\item Clear console: \R{\T{clear}}
	\item Exit from bash terminal: \R{\T{logout}}
\end{enumerate}

\section*{Git commands}

\begin{enumerate}
	\item Check git version: \R{\T{git --version}}
	\item Initialize a new empty local repository: \R{\T{git init}} :- a hidden folder will be created with name ".git"

	\item Adding/Staging(give access to git tracker) files:
		\begin{enumerate}
			\item Specified: \R{\T{git add fileName}}
			\item Multiple: \R{\T{git add file1 file2 file3}}, space separated.
			\item Specified by extension: \R{\T{git add *.ext}}
			\item All: \R{\T{git add .}}
		\end{enumerate}
	\item Staging logs:
		\begin{enumerate}
			\item Full log: \R{\T{git log}}, press 'q' to quit from long log list. Each log come up with a unique hexadecimal ID(String with 40 characters maybe). Each commit contains a complete snapshoot of working directory. \R{log} is the history the snapshoots of a repository.
			\item Press space for move to the next page of log.
			\item One lined: \R{\T{git log --oneline}}
			\item Last N commits by a non negative integer: \R{\T{git log -N}}, N means non-negative integer as, 2, 6. eg. \R{\T{git commit -3}} will show last 3 log.
			\item In reverse order, full: \R{\T{git log --reverse}}.
			\item In reverse order, one lined: \R{\T{git log --oneline --reverse}}.
			\item Show changed files in each commit: \R{\T{git log --stat}} or \R{\T{git log --oneline --stat}}
			\item See actual changes in each commit: \R{\T{git log --patch}} or \R{\T{git log --oneline --patch}}
			\item Show commit changes: \R{\T{git show commitID}} or, \R{\T{git show HEAD~n}}, here HEAD is the last commit we did and n is an integer indicating that go n steps back from HEAD and then show the commit changes.
			\item Filter logs to be shown by
				\begin{enumerate}
					\item Author: \R{\T{git log --author="author\_name"}}
					\item Date
					\item Before with this date: \R{\T{git log --before="yyyy-mm-dd"}}
					\item After with this date: \R{\T{git log --after="yyyy-mm-dd"}}
					\item Indication: \R{\T{git log --before/after="yesterday/one week ago/two week ago/one month ago/two month ago"}}
					\item the commit messages those has the string/substring "WORD". It is case sensitive. : \R{\T{git log --grep="WORD"}}
				\end{enumerate}
		\end{enumerate}
	\item File list in staging area: \R{\T{git ls-files}}
	\item File list with tree: \R{\T{git ls-tree commitID}}	or, \R{\T{git ls-tree HEAD~n}}
	
	\item Status:
		\begin{enumerate}
			\item Full status: \R{\T{git status}}
			\item Short status: \R{\T{git status -s}}, I think, \R{\T{-s}} stands for "short".
		\end{enumerate}
		
	\item Configuration:
		\begin{enumerate}
			\item Full: \R{\T{git config --list}}
			\item Specified by key: \R{\T{git config <key>}} e.g. To get user name that already been configured, \textcolor{red!70}{{git config user.name}}
			\item \R{\T{git config --global user.name "My Name"}}
			\item \R{\T{git config --email myEmail}}
		\end{enumerate}
	\item Get a full copy of an existing repository: \R{\T{git clone <url>}}
	
	\item Set default editor for git
	\item Open the dot git folder: \R{\T{start .git}}
	\item Remove tracking access from git: delete the \R{\T{.git}} hidden directory.
	
	\item Update git: \R{\T{git update-git-for-windows}}
	\item Every git commit contains a distinct ID, Message, Date/Time, Author, Complete snapshot of project.

	\item Remote:
		\begin{enumerate}
			\item Remote repository is a repository exits on server/online.
			\item Add: \R{\T{git remote add <shortname> <url>}}
			\item Lists all the remotes with server link details: \R{\T{git remote -v}}
			\item Inspect: If you want to see more information about a particular remote: \R{\T{git remote show <remoteShortName>}}
			\item Rename: \R{\T{git remote rename <oldShortName> <newShortName>}}
			\item Remove: \R{\T{git remote rm <shortname>}} or \R{\T{git remote remove <shortname>}}
			\item To see remote servers you have configured, you can run the \R{\T{git remote}} command. It lists the shortnames of each remote handle you've specified. If you've cloned your repository, you should at least see "origin" that is the default name Git gives to the server you cloned from. Can use \R{\T{-v}} option to shows Related URLs that git has stored for the shortname.
		\end{enumerate}	
	
	\item \textbf{Have Bug}, After adding a remote to a repository, we can fetch by the shortname: \R{\T{git fetch <shortname>}}
\end{enumerate}

\section*{Kali Linux}
\begin{enumerate}
	\item Kali Linux is a Linux distro based on debian specially made for hackers cause kali has a lot of hacking tools and penetration testing tools preinstalled and preconfigured in it.
	\item We use python for ethical hacking and will make many many tools with this. Automate.
	\item Know who is the user: whoami
	\item Internet configuration: ifconfig
	\item Three must have skill: gathering information, 	
	
	\item \R{\T{cat}} \textit{(concatenate)} command
		\begin{enumerate}
			\item Show all the contents of given  filename: \R{\T{cat file\_name}}
			\item Show contents of multiple files: \R{\T{cat file1 file2}}
			\item Show contents preceding with line numbers: \R{\T{cat -n file\_name}}
			\item Create a file: \R{\T{cat > file\_name.ext}}
			\item Copy content of file1 to file2: \R{\T{cat file1 > file2}}
			\item : \R{\T{cat }}
			\item : \R{\T{cat }}
		\end{enumerate}
\end{enumerate}

\subsection{Password Cracking}
Passwords are stored inside a server in a hash(crazy, massive numbers and letters) form, not in original plain text from.
Options: Rainbow table, dictionary attack, brute force method, hashcat, \\
For brute force attack we use hydra that are preinstalled on Kali Linux.

\subsection*{Sherlock}
Sherlock is a CLI based python program that can find all social media accounts by a username that provided as a parameter. The name taken from famous detectives name: "Sherlock Homes" (I like the guy, yeah!). Source link on github: %\href{GITHUB-Sherlock}{https://github.com/sherlock-project/sherlock} . The full documentation and installation process described on that site. Or you can open it on "Google Cloud Shell" or run it on "repl.it" .

\subsection*{Blackeye: The ultimate phishing tool for free} % \R{\T{}}
Blackeye is a great phishing tool for hackers. Here is the complete instruction:
Before this: Create and connect your \textbf{ngrok} account through terminal to get localhost (one for free). Or if it already exists on your system then kill the previous process by this command: \R{\T{kill -9 {PID\footnote{Process ID}}}} (You need to find out that wheres the process is running, may be it on the sites folder). To get the PID run the command: \R{\T{top}} and find PID of \textbf{ngrok}. \R{\T{ctrl + c}} to get out from the process.
\begin{enumerate}
	\item Get blackeye:  \R{\T{ git clone https://github.com/An0nUD4Y/blackeye.git}}
	\item Change directory to blackeye: \R{\T{cd blackeye}}
	\item Run blackeye.sh:
		\begin{enumerate}
			\item \R{\T{bash blackeye.sh\footnote{better choice}}}
			\item \R{\T{chmod +x blackeye.sh}} $\rightarrow$ \R{\T{./blackeye.sh}}
		\end{enumerate}
	\item Choose an option
	\item While the process is running, do this job: open two terminal, one on blackeye, second on that sites folder that you wanna phish. Run this command on first terminal: \R{\T{./ngrok http 8080}}, before this, run this command on second terminal: \R{\T{php -S localhost:8080}} .
	\item Now at the first terminal, the phishing link is ready to go\dots
	\item Wait for victim to response.
	\item After loging into the phishing site. It will take the victim to original page.
\end{enumerate}

\vspace{5mm}

Basically, blackeye tool is owned by \textbf{thelinuxchoice}. \\
Cautions:
\begin{enumerate}
	\item Become anonymous, zsecurity has a tutorial on youtube for this.
	\item Change the mac address:
		\begin{enumerate}
			\item \R{\T{ifconfig eth0 down}} to disable ethernet connection eth0
			\item \R{\T{}}
			\item \R{\T{ifconfig eth0 up}}
		\end{enumerate}
	\item Get connected with a VPN.
	\item Attack with the help of Social Engineering as: \texttt{Hey, it's an important message for you. Please check it out right now.} Or, \texttt{Check your facebook security and make it strong.}, etc. Also can use my various corporate g-mail accounts for this. Create a g-mail from non-native country with the same name as company or site like, \textit{facebooksecurityhub3032@gmail.com}
	\item Create another git and github account for hacking.
	\item Modifying the phishing \textit{website.html} from blackeye/sites directory would be good to phishing attack. Learn \textbf{HTML}, \textbf{CSS} and \textbf{PHP} well. Or insect the code from original page with the help of browser insect option.
	\item 
\end{enumerate}

This tutorial pdf is for educational purposes only. Do not hack anyone without permission. \\

Some issues: \textit{how to fix blackeye tool for phishing website link is not showing issue, how to fix err\_ngrok\_108 issue, what is local host and domain, what is dns server, what is ubuntu server and why should we use it, what is dns poisoning.}

\section*{Mail: I will use for educational hacking}

\section*{Compile and Execute C Program through CMD: Windows OS}
\begin{enumerate}
	\item Check if gcc compiler available: \R{\T{gcc -v}}
	\item \R{\T{cd Drive\_name: full\textbackslash path}} to that folder that has the required c source file
	\item Compile: \R{\T{gcc c\_program\_name.c}}
	\item An executable .exe file will be created at the same directory, just run this.
\end{enumerate}

\section*{Manim}
Manim stands for \textbf{Mathematical Animation}. Created by \textbf{3blue1brown}. Manim is a custom python library used to animate math explanation. Manim runs on python 3.6 or higher but 3.8 is recommended. \href{https://docs.manim.community/en/stable/}{\textcolor{red}{DOCS here}}.\\
\textbf{Mobjects} class is the root abstract class in manim. All other \textbf{Mobjects} like Circle, Rectangle, Triangles are derived from this class.\\

\begin{center}
\begin{forest}
	[Mobject
		[Circle]
		[Rectangle]
		[Triangle]
		[Arrow]
		[Axes]
		[FunctionGraph]
		[Barchart]
	]
\end{forest}
\end{center}

\begin{enumerate}
	\item \R{\T{manim -flags file\_name.py class\_name}}
	\item Flags
			  \begin{enumerate}
			  	\item \R{\T{p - play when class rendered}}
			  	\item \R{\T{ql - low quality}}
			  	\item \R{\T{qm - medium quality}}
			  	\item \R{\T{qh - high quality}}
			  	\item \R{\T{qk - 4k quality}}
			  	\item \R{\T{a - render all scene class}}
			  	\item \R{\T{f - open file browser location}}
			  	\item \R{\T{i - .gif instead of .mp4}}
			  \end{enumerate}
	\item Animation time by run\_time: \R{\T{self.play(animation(mobject), run\_time = seconds)}}
\end{enumerate}

\section*{SQLite}
	\justify
	{
	SQLite is a lightweight, reliable, portable database management tool, written in C, published in $2000$. Every commands(query in SQL) ends with a semicolon. We can write query in both uppercase or lowercase but writing queries in uppercase is conventional. TEXT type data allowed with both of single quote and double quote. Most like Python syntax/keywords. Not case-sensitive or case-insensitive.\\
	}
	Data types in SQLite:
	\begin{enumerate}
		\item NULL
		\item INTEGER
		\item REAL
		\item TEXT
		\item BLOB
	\end{enumerate}
	SQLite has no built-in boolean types. So, we have to store boolean 0(False) \& 1(True) inside of a INTEGER variable.\\

	\pagebreak
	Queries:
	\begin{enumerate}
		\item Start query operation on a SQL file with sqlite3 engine:\\
			\R{\T{sqlite3 file\_name.sql}}\\
			e.g. \textcolor{red!70}{sqlite3 flights.sql}
		\item Get help: \R{\T{.help}}
		\item Create table: \\
			\R{\T{create table table\_name (column\_name1 data\_type optional\_properties, column\_name2 \dots);}}\\
			e.g. \textcolor{red!70}{create table flights (id INTEGER PRIMARY KEY AUTOINCREMENT, origin TEXT NOT NULL, destination TEXT NOT NULL, duration INTEGER NOT NULL);}
		\item Show all the tables exist inside the database: \R{\T{.tables}}
		\item Insert data into table:\\
			\R{\T{INSERT INTO table\_name (comma separate columns) values(comma separated values for each corresponding column);}}\\
			e.g. \textcolor{red!70}{INSERT INTO flights (origin, destination, duration) values('New York', 'San Francisco', 415);}
		
		\item Get all data from a specific table in a \textbf{SQL} file: \R{\T{select * from table\_name;}}
			\begin{enumerate}
				\item \R{\T{select}}: get/obtain/return.
				\item \R{\T{*}}: all columns.
				\item \R{\T{from}}: from which table.
				\item \R{\T{table\_name}}: specify the table in a particular SQL file.
			\end{enumerate}
			
		\item Organize table in a specific mode: \R{\T{.mode value}}\\
		Values would be:
		\begin{enumerate}
			\item \R{\T{column}}:  Preferred by sir \href{https://brianyu.me/}{Brain Yu}.
			\item \R{\T{ html}}: Then we can copy and paste the tags and elements in a HTML document.
			\item \R{\T{box}}: Boxed table, so pretty as well. My preferred.
			\item \R{\T{line}}:  Good one.
			\item \R{\T{table}}
			\item \R{\T{insert}}: With SQL insert statement.
			\item Or get description: \R{\T{.help .mode}}
		\end{enumerate}
		
		\item Select all columns where rows have a particular value:\\
			\R{\T{select * from table\_name where column\_name = value;}}\\
			e.g. \textcolor{red!70}{\T{select * from flights where origin = 'Tokyo';}}
		\item Select some columns: \R{\T{SELECT column1, column2,\dots\hspace{0.1cm} FROM table\_name;}}
			e.g. \textcolor{red!70}{\T{SELECT origin, destination FROM flights;}}
		\item Conditional selection:\\
			\R{\T{SELECT * FROM table\_name WHERE conditions}}(python's like).\\
			e.g. \textcolor{red!70}{\T{SELECT * FROM flights WHERE duration > 300;}}\\
			e.g. \textcolor{red!70}{\T{SELECT * FROM flights WHERE duration < 200 and origin = 'New York';}}\\
			e.g. \textcolor{red!70}{\T{SELECT * FROM flights WHERE duration > 300 or destination = 'San Francisco';}}\\
			e.g. \textcolor{red!70}{\T{SELECT * FROM flights WHERE origin IN ('New York', 'San Francisco');}}
		\item Select by pattern matching: \R{\T{SELECT * FROM flights WHERE origin like '\%regex\%'}}\\
			e.g. \textcolor{red!70}{\T{SELECT * FROM flights WHERE origin LIKE '\%a\%';}}
		\item Delete a row: \R{\T{DELETE FROM table\_name WHERE conditions;}}\\
			e.g. \textcolor{red!70}{\T{DELETE FROM flights WHERE duration > 100;}}
		\item Delete entire table: \R{\T{DROP TABLE table\_name;}}
			e.g. \textcolor{red!70}{\T{DROP TABLE flights;}}
		\item Exit from sqlite3 prompt console: \R{\T{.quit}}
	\end{enumerate}

\end{document}
