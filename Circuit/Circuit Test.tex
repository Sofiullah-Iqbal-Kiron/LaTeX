\documentclass[11pt]{article}

\usepackage[american]{circuitikz} % Circuit based on tikz. TikZ will loaded automatically. circuitikz commands are just TikZ commands.

\title{Circuit Test}
\author{Sofiullah Iqbal Kiron}
\date{26 August, 2021: 12:22 AM}

\begin{document}

\maketitle

Starting with circuitikz. TikZ will loaded automatically. circuitikz commands are just TikZ commands. It is good practice keep circuits in a figure. [element=value]\\
\begin{figure}
    \tikz \draw (0, 0) to[R=$R_1$] (2, 0); % [element=value]
    \centering
\end{figure}

\begin{circuitikz}[]
    \draw (0,0) to[isource] (0,3) -- (2,3)
    to[R] (2,0) -- (0,0);
\end{circuitikz}


\section*{Block Diagram of 1-bit Register}
% Flip-Flip definition
\tikzset
{sr-ff/.style={flipflop, flipflop def={t1=S, t2=CP, t3=R, t4={\ctikztextnot{Q}}, t6=Q, nd=1}}}
\begin{figure}
    \begin{circuitikz}
        \draw (0,0) node[flipflop D];
    \end{circuitikz}
    \centering
    \caption[]{Diagram of 1-bit register.}
\end{figure}


\end{document}