\documentclass[11pt]{book}


\usepackage{csquotes}
\usepackage{enumitem} % Change itemize bullet/label
\usepackage{amssymb} % That $\blacksquare$ symbol belongs to this American Math Society Symbol package.

\begin{document}

\part{Business}

\chapter{Introduction to Business}


\section{What is Business}
Business is the activity performed by an individual or a group of people with a view to earn profit within the government rules and regulations. Main purpose of business is to earn profit.

\section{Elements of Business}
Three ($3$) main components of business:
\begin{enumerate}
	\item[Trade:] is the final stage of business activity and it involves sales and purchase of goods. It can divided into two types such as:
		\begin{enumerate}
			\item Home trade: when trade takes place within the national boundaries.
			\item Foreign trade: when trade takes place across the national boundaries.
		\end{enumerate}
	\item[Commerce:] It is the process of buying, selling and other activities which facilitate trade such as storing, packaging, transportation, insurance, banking, finance and marketing promotion. The principle function of commerce is to remove the hindrances of person, place, time, exchange and knowledge.
	\item[Industry:] Industry refers to that part of business activity which is engaged in rising, producing, processing, fabrication, extracting and conversion of goods.
\end{enumerate}

\section{Features/Characteristics of Business}
\begin{itemize}[label={$\blacksquare$}]
	\item Profit
	\item Risk, it is possible that there may be loss instead of gain. There is no business activity which is not subject to loss.
	\item Legality, illegal business is not be considered as business.
	\item Creation of utilities
	\item Exchange of goods and services is the foundation function of business. before its existence, business doesn't formed.
	\item Forecasting
	\item Rendering service to the society, the responsibility of businessman is to deliver the goods and services in way that is beneficial to the society.
\end{itemize}

\section{Objective of business}
Success in business cannot be achieved without setting right objectives. Some objectives are social welfare oriented and some of are profit oriented.
\begin{itemize}
	\item Profit, the main objective of business to earn it.
	\item Business Profit = Expenses - Revenue
	\item Economic Profit
	\item Survival, prime objective of business.
	\item Growth, is inevitable for a firm to be successful.
	\item Social Responsibility: the responsibility of a businessmen is to supply goods and services in that way which are not harmful to the society. Only profit earning cannot be the sole motive of business activity. Another responsibility of a businessman is to supply goods and services at a fair price.
\end{itemize}

\section{Plant, Firm \& Industry}
\subsection{Plant}
The term plant refers to a place or establishment where goods are produced. This includes not only building and machinery but also the workers employed therein.
\subsection{Firm}


\section{Business Input and Output}
Input:
\begin{itemize}
	\item Human resource
	\item Capital
	\item Managerial resource
	\item Technological resource
\end{itemize}
Output:
\begin{itemize}
	\item Product
	\item Service
	\item Profit
	\item Satisfaction
	\item Goal
\end{itemize}

\chapter{Sole Proprietorship}
\enquote{Sole} means single and \enquote{proprietorship} means ownership. So, the business organization in which a single person owns, manages and controls all the activities of the business is known as sole proprietorship form of business organization. Or, a business enterprise exclusively owned, managed and controlled by a single person with all authority, responsibility and risk-reward is called a sole proprietorship business. \\
Characteristics or Features of sole proprietorship business:
\begin{itemize}
	\item Easy formation
	\item Single ownership
	\item No sharing of profit-loss
	\item Limited capital
	\item Full controlling power only goes to the sole.
	\item Unlimited liability
	\item Limited area of operation
	\item Freedom in selection of trade, so need to consult any person for trade.
	\item Secrecy
	\item Personal relations
\end{itemize}
Advantages of sole proprietorship business:
\begin{itemize}
	\item Easy formation.
	\item Direct motivation.
	\item Quick decision and prompt action.
	\item Better control.
	\item Maintenance of business secrecy.
	\item Personal relations.
	\item Flexibility in operations.
	\item Encourages self-employment.
	\item Minimum government regulations.
\end{itemize}
Disadvantages of sole proprietorship business:
\begin{itemize}
	\item Limited capital.
	\item Unlimited liability.
	\item Lack of continuity.
	\item Limited size.
	\item Lack of managerial expertise.
	\item Limited scope of employees.
	\item Limited scope of expansion.
	\item Risk of wrong decisions.
\end{itemize}

\section{Customer vs Consumer}
\begin{enumerate}
	\item Customer is one who purchasing the goods. - Consumer is the one who is the end user of any goods or services.
	\item Customer has ability to resell wheres cousumer does not hold this property.
	\item Customer need to purchase a product or service in order to use it. - For a consumer purchasing a product or service is not essential.
	\item Customer buy for resell or consumption wheres consumer buy only for consumption.
	\item Customer must need to make payment wheres consumer is not forced to make payment.
	\item Customer can be individual or company but consumer is individual, group or family.
\end{enumerate}

\part{Management}
Managing is universal and management is everywhere. Management is getting things done through others. It is basically a process of planning, organizing, staffing, leading and controlling. \\
\textbf{Management is the process of designing and maintaining an environment in which individuals, working together in groups, efficiently accomplish selected aims.} \\
The main purpose of management is to increase productivity; this implies effectiveness and efficiency. \\
\textbf{Productivity: } can be defined as the input-output ratio within a time period with due consideration for quality. \\
\textbf{Effectiveness: } is the achievement of objective or reading to the goal. Making the right decisions and successfully implementing them. \\
\textbf{Efficiency: } is the achievement of end with least amount of resources.

\section{Functions}
There are four basic functions of management:
\begin{enumerate}
	\item Planning: projected course of actions.
	\item Organizing: co-ordinating activities and resources.
	\item Leading: motivating and managing people.
	\item Controlling: monitoring and evaluating activities.
\end{enumerate}

\end{document}