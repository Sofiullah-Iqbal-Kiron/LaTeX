\documentclass[11 pt]{book}


%\usepackage[utf8]{inputenc} % Input encoder, basically for English
\usepackage[table]{xcolor} % Color your document
\usepackage[skins]{tcolorbox} % Colored / nice textbox
\usepackage[document]{ragged2e} % Various text alignment.
\usepackage[short, nodayofweek, level, 12hr]{datetime}
\usepackage[usestackEOL]{stackengine}
\usepackage[left=3cm, right=3cm, top=3cm, bottom=2cm]{geometry}

\usepackage{multicol} % Distribute paragraph in muticolumn
\columnseprule = 0.5 mm % Column separator rule width

\usepackage
{
 array, % Manipulate the table rules
 listings, % Code listing
 graphicx, % Picture add and design with graphics
 authblk, % Author affiliation
 tikz, % Graphical plot
 pgfplots, % Graph plots
 tabularx, % More flexible formatted table
 lipsum, % Give lipsum article
 mVersion, % Adding version later of date
 framed, % For make a frame to CodeListing or any place.
 longtable, % Create and break out large / long table
 wrapfig, % Wrap the figure with text
 fontawesome, % Print free icons form font awesome.
 import % Add external (.tex)files in this doc.
}

\usepackage[hidelinks]{hyperref} % Add hyperlink / clickable link


\setVersion{0.0}
\increaseBuild % Will update version at each recompilation


\title{Programming Document}
\author
{
	Sofiullah Iqbal Kiron\\
	\href{mailto:sofiul.k.1023@gmail.com}{sofiul.k.1023@gmail.com}
}
\date{11 April, 2020 \\ \currenttime \\ Version: \version}
\affil
{
	BSMRSTU, Department of CSE \\
	SHIICT \\
	{\tiny Copyright\faCopyright\hspace{2pt} under Sofiullah Book Agency}
}
% \institute{Bangabandhu Sheikh Mujibur Rahman Science and Technology University}(Omit this, replaced by \affil{})

% Setting up code listing environment properties.
\lstset
{
 language = C++,
 backgroundcolor = \color{black!5},
 basicstyle = \footnotesize\ttfamily, % Add all style properties without comma separator.
 keywordstyle = \color{blue},
 commentstyle = \color{green!80},
 showstringspaces = false,
 stringstyle = \color{red},
 captionpos = b %, b: bottom
 % numbers=left/right
}


% Setting up link properties.
\hypersetup
{
	colorlinks = true,
	linkcolor = red,
	filecolor = magenta,
	urlcolor = cyan,
	pdftitle = {Wanna be a programmer?},
	bookmarks = true,
	hyperindex = true, % Makes the page numbers of index entries into hyperlinks
	linktocpage = false, % Makes the page numbers instead of the text to be link in the Table of Contents
	breaklinks = false, % Allows links to be broken into multiple lines
	frenchlinks = false, % Use small caps instead of colours for links
}


\renewcommand\contentsname{A-Z Menu}
\renewcommand\chaptername{Unit}


\begin{document}

\frontmatter % \mainmatter, \backmatter are used for book class specially.

\pagecolor{green!90}
\maketitle
\pagecolor{white}
\tableofcontents
\listoffigures
\listoftables
\justify
{
	\raggedleft\vfill\itshape\Longstack[l] %You can change this properties to add more formation. As, \Longstack[l], \hfill, \raggedright, not \itshape, etc. (Style)
		{
			Thank You,\\
			Sofiullaha Iqbal Kiron.
		}\par
}

\mainmatter

% Chapter One (1).
\chapter{Quotes}
\import{C:/Users/Hp/Documents/LaTeX/Programming Document/Files to Import}{C:/Users/Hp/Documents/LaTeX/Programming Document/Files to Import/Quotes.tex}

% Chapter Two (2).
\chapter{C++, Problems and Algorithm}
\import{C:/Users/Hp/Documents/LaTeX/Programming Document/Files to Import}{C:/Users/Hp/Documents/LaTeX/Programming Document/Files to Import/C++, Problems and Algorithm.tex}

% Chapter Three (3).
\chapter{Java}
\import{C:/Users/Hp/Documents/LaTeX/Programming Document/Files to Import}{C:/Users/Hp/Documents/LaTeX/Programming Document/Files to Import/Java.tex}

% Chapter Four (4).
\chapter{JavaScript}
\import{C:/Users/Hp/Documents/LaTeX/Programming Document/Files to Import}{C:/Users/Hp/Documents/LaTeX/Programming Document/Files to Import/Javascript.tex}

% Chapter Five (5).
\chapter{Julia}
\import{C:/Users/Hp/Documents/LaTeX/Programming Document/Files to Import}{C:/Users/Hp/Documents/LaTeX/Programming Document/Files to Import/Julia.tex}

% Chapter Six (6).
\chapter{Coursera Algorithm}
\begin{itemize}
	\item[Algorithm: ] Method for solving a problem.
	\item[Data Structure: ] Method to store information.
\end{itemize}
Algorithms of all of this course are implemented in Java and I love Java. 

% Chapter Seven (7).
\chapter{Windows Command Prompt}
Widows command prompt or windows terminal is a console for creating command line. If you wanna be a great developer then you must get used to uses of this.\\
\begin{center}
	\begin{tabularx}{0.9\textwidth}
	{|>{\raggedright\arraybackslash}X
	 |>{\raggedleft\arraybackslash}X	
	|}
		\hline
		{\color{red}Task}& {\color{red}Command Code}\\
		\hline
		\hline
		For clear total screen & cls\\
		\hline
		Will show all color code & color/?\\
		\hline
		For start a soft program & start soft\_name\\
		\hline
		Wanna exit from current folder? & exit\\
		\hline
		Open a file from current folder & cd folder\_name\\
		\hline
		View name of all files in current folder & dir\\
		\hline
		View name of all files in current folder with hidden files & dir/a\\
		\hline
		Make a new folder & mkdir folder\_name\\
		\hline
		Back from current folder & cd ..\\
		\hline
		Back more than one & cd ../.. as this\\
		\hline
		Remove directory or folder(If the folder is currently empty) & rmdir folder\_name\\
		\hline
		Remove directory or folder(If not empty) & rmdir /s folder\_name and then y or YES confirmation.\\
		\hline
		Compile java source code & javac file\_name.java\\
		\hline
		Run java class & java file\_name\\
		\hline
		Type few first words of file and then automatically filled full name & Press tab key :-not command\\
		\hline
		Delete all file as same type & del *.extension\\
		\hline
		
	\end{tabularx}
\end{center}

% Chapter Eight (8).
\chapter{Type Setting}
\import{C:/Users/Hp/Documents/LaTeX/Programming Document/Files to Import}{C:/Users/Hp/Documents/LaTeX/Programming Document/Files to Import/Typesetting.tex}

\chapter{Import Practice}
Not imported.

\backmatter

\begin{center}
	\begin{tcolorbox}[enhanced,
		size=minimal, auto outer arc,
		width=2.1cm, octogon arc,
		colback=red, colframe=white, colupper=white,
		fontupper=\fontsize{7mm}{7mm}\selectfont\bfseries\sffamily,
		halign=center, valign=center,
		square,arc is angular,
		borderline={0.2mm}{-1mm}{blue!70!green}]
		OVER
	\end{tcolorbox}
\end{center}

\end{document}
