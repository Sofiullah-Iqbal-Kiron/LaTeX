\documentclass[14 pt, letterpaper]{extarticle}

\usepackage[document]{ragged2e}
\usepackage
{
	xcolor,
	graphicx, % \includegraphics and graphics related
	hyperref,
	extsizes, % increase extarticle size manually
	csquotes
}

\title{Cyber and Intellectual Property Law}
\author
{
	\includegraphics[scale=0.2]{../Commands/User Profile} \\ % scale is better scaling for picture. Scale 0.2 means 20 percent will be the printed size of main size.
	\textit{Sofiullah Iqbal Kiron} \\
	\rule{11 cm}{2 pt}
}
\date{21 may, 2022 \\ {\tiny Last compiled: \today}}

\begin{document}
	
	\maketitle

	\section{Definition Nature}
	Cyber law is a law related to IT and ICT. Internet has rendered a tremendous prospective aspect for the human civilization. The inventors of internet may not really anticipate the scope and far reaching consequences of cyberspace. Cyberlaw is a law governing computer and internet.\linebreak\linebreak
	Cyber crime cases:
	\begin{enumerate}
		\item Online banking frauds.
		\item Online share trading frauds.
		\item Source code theft.
		\item Credit card fraud.
		\item Tax evasion.
		\item Virus attack.
		\item Cyber sabotage.
		\item Email hijacking.
		\item Denial of service.
		\item Hacking.
		\item Pornography.
		\item Defamation.
		\item Selling illegal products.
	\end{enumerate}

	
	\textbf{The law related to control the behavior, rights, duties and obligations of the benificiaries of satellite station, satellite communication, internet, cyberspace, ICT etc. is termed as Cyberlaw.}
	
	\textbf{Cyberspace is the metaphorical space of computer systems and computer networks where electronic data is stored and online communication take places.}
	
	Nature of cyber laws:
	\begin{enumerate}
		\item Jurisdiction and sovernity of internet.
		\item Net Neutrality.
		\item Free space in cyberspace.
		\item Internet regulation in different countries.
	\end{enumerate}
	
	\section{Cyber Crime}
	\textbf{Cyber crime may be defined as E-Crime that are almost conventional crimes in nature committed by using computer \& ICT with an intention to make social disorder.}
	\linebreak
	Types of cyber crime:
	\begin{enumerate}
		\item Financial Crimes.
		\item Cyber Pornography.
		\item Sale of illegal articles.
		\item Online gambling.
		\item Intellectual property crimes.
		\item Email Spoofing.
		\item Forgery.
		\item Cyber Defamation.
		\item Cyeber Stalking.
	\end{enumerate}
	
	\section{Jurisdiction and Cyber Crime}
	The internet does not tend to make geographical and jurisdictional boundaries but the internet users are remaining under physical jurisdiction. \\
	Jurisdiction is an aspect of state sovereignty. If refers to judicial, legislative and administrative competence.
	So the cyber jurisdiction is the jurisdiction enforced by a state upon it's citizen who uses cyberspace.
	
	\section{Question solve}
	Intellectual Property(abbreviated IP) refers to creations of the mind, such as inventions; literary and artistic works, designs and symbols, names and images used in commerce. IP is protected in law. \\
	The four ($4$) main types of intellectual property are
	
	\begin{enumerate}
		\item[Patent:] is a property right for an investor that is typically granted by a government agency. The patent allows the inventor exclusive rights to invention, which could be a design, process, an improvement or physical invention such as a machine.
		\item [Copyright:] provide authors and creators of original material the exclusive right to use, copy or duplicate their material.
		\item[Trademark:] is a symbol, phrase or insignia that is recognizable and represents a product that legally separates it from other products. A trademark is exclusively assigned to a company, meaning the company owns the trademark so that no others may use or copy it.
		\item [Trade secret:] is a company's process or practice that is not public information, which provides an economic benefit or advantage to the company or holder of the trade secret.
	\end{enumerate}
	
	\textbf{Short notes}\linebreak
	\begin{itemize}
		\item Passing off happens when someone deliberately or unintentionally passes of their goods or services as those belonging to another party. This action of misrepresentation often damages the goodwill of a person or business, causing financial or reputation damage. The elements of passing off is goodwill, misrepresentation and damage.
		\item Digital Signature in cryptography, a digital signature or digital signature scheme is used to pretend the security properties of a signature in digital form. Digital signature schemes normally give two algorithms:
		\begin{enumerate}
			\item One for signing which involves the user's public key.
			\item Other for verifying signatures which involves the user's public key.
		\end{enumerate}
		The output of the signature process is called the digital signature. Digital signatures are used to create Public Key Infrastructure(PKI) schemes. User's public key is tied to user by a digital identity certificate issued by a certificate authority.
		\item Deceptive Similarity is an issue of  marks can be defined as similarity between trademarks that can lead the general public of average intelligence to believe that the mark in question is somehow related to a registered or well-known trademark. So, a trademark should not be registered if it is deceptively similar.
		\item Infringement of copyright or copyright violations occur when an authorized party recreates all or a portion of an original work, such as a work of art, music or a novel. The duplicated content need not be an exact replica of the original to qualify as an infringement.
	\end{itemize}
	
	\begin{justify}
		
	\section{Hacking and punishment according to ICT act. 2006}
	\textbf{Hacking:} is harming any computer, server, network or any other electronic system by accessing it unlawfully with some techniques and special computer programs.
	\textbf{Punishment:} Whoever commits hacking shall be punished with imprisonment of either description for a term which may extend to three years, or with fine which may extend to Taka one crore or with both.
	
	\section{Objectives of CyberLaw}
	The objectives of the ICT Act, 2006 has been provided following purposes such as To smooth the progress of electronic filing of documents with government agencies and statutory corporations and to promote efficient delivery of government services by means of reliable electronic records. To help to establish uniformity of rules, regulations and standards regarding the authentication and integrity of electronic records. To facilitate electronic commerce, eliminate barriers to electronic commerce resulting from uncertainties over writing and signature requirements, and to promote the development of the legal and business infrastructure necessary to implement secure electronic commerce and so many others objectives have been included here.
	
	\section{Advantages of ICT act 2006}
	This Act has some disadvantages and also some advantages. This Act has provided us few advantages like as under the ICT Act, 2006, conduct important issues of security, which are so critical to the success of electronic transactions. The Act has given a legal definition to the concept of secure digital signatures that would be required to have been passed through a system of a security procedure, as stipulated by the government at a later date. On the other hand Companies now be able to carry out electronic commerce using the legal infrastructure provided by the Act. Subsequently this Act provided other facilities to run cyber or Information and Technology business.
	
	\section*{Question Solve 2020}
	
	\subsection*{1}
	\subsubsection*{a: What is meant by intellectual property?}
		Intellectual Property (IP) refers to creations of mind such as inventions, literary and artistic works, designs and symbols, names and images used in commerce.
		
	\subsection*{5}
	\subsubsection*{a: Procedure of registration of trademark.}
		\begin{enumerate}
			\item Trademark search.
			\item Filing trademark application.
			\item Examination.
			\item Publication.
			\item Registration Certificate.
			\item Renewal.
		\end{enumerate}
	\subsubsection*{b: Define distinguish between passing off and infringement.}
	\textbf{Trademark:} may be defined as a lawfully protected abstract, word, symbol, color, mark, slogan or a mixture of those related to an organization or a selected product that differentiates it from other accessible within the market. OR:- Trademark means a mark capable of being represented graphically and which is capable of distinguishing the goods or services of one person from those of others and may include the shape of goods, their packaging and combination of colours.
	\textbf{Passing off:} is \enquote{making some false representation likely to induce a person to believe that the goods or services are those of another.}
	\linebreak
	\textbf{Infringement:} Trademark Infringement is when a person makes unauthorized use of a trademark or service mark. As stated above the law of passing off is for unregistered trademark whereas trademark infringement is for the registered trademarks.
	\begin{enumerate}
		\item Trademark provides protection to registered goods and services whereas Passing Off provides protection to unregistered goods and services. This is one of the most important differences between Passing Off and trademark infringement. But the point to be noted here is that the remedy provided in both Passing Off and trademark infringement is the same.
		\item The other difference between Passing Off and trademark infringement is that in Passing off it is not essential for the defendant to use the trademark of the plaintiff to bring an action of passing off but in trademark infringement, it is not the case.
		\item In the case of trademark infringement, the burden of proof lies on the plaintiff.
		\item Passing off is a common law remedy whereas Trademark infringement is a statutory remedy.
		\item For trademark infringement prosecution under criminal remedy is quite easy as compared in the case of Passing off.
		\item For trademark infringement registration is essential whereas for passing off Goodwill, damage, misrepresentation is essential.
	\end{enumerate}
		
	\subsection*{6}
	\subsubsection*{a: Define Cybercrime}
		Cybercrime may be defined as E-Crime. E-Crime covers offences where a computer or other information and communication technology are used to commit an offence. E-Crime is a type of offence specifically related to computer. \textbf{E-Crime are almost conventional crime in nature committed by using computer and ICT with an intention to make social disorder.}
	\subsubsection*{b: Write down different kinds of cybercrime}
		\begin{enumerate}
			\item Financial Crimes: include cheating, credit card frauds, money laundering.
			\item Cyber Pornography includes pornographic websites; publish and print pornographic magazines by using computers and the internet to download and transmit pornographic pictures, photos; writings.
			\item Sale of illegal articles include sale of narcotics, weapons and wildlife, illegal medicine. This can be by posting information on websites, auction websites and bulletin boards or simply by using email communication. Many of the auction sites are believed to be selling cocaine in the name of honey. In Bangladesh, it is also practicing.
			\item Online gambling: many websites offer online gambling. Many of these websites may actually be fronts for money laundering. Cases of hawala transactions and money laundering over the internet have been reported. These sites may have any relationship with drug trafficking. This is yet to be explored.
			\item Intellectual property crimes include software piracy, copyright infringement, trademarks violations, theft of computer source code. Some cyber squatters are registering domain with Network solutions under different fictitious names. Transfer of domain names to any third party should be restricted.
			\item Email spoofing: A spoofed E-Mail is that E-Mail which appears to originate from one source but actually has been sent from another source.
			\item Forgery: Counterfeit currency notes, postage and revenue stamps, mark sheets, certificate etc can be forged using sophisticated computers, printers and scanners. This is becoming a booming business nowdays.
			\item Cyber defamation: This occurs when defamation takes place with help of computers or the internet. For example, someone publishes defamatory matter about someone on a website or other internet based space.
			\item Cyber stalking: is \enquote{pursuing stealthily}. Cyber stalking involves a person's movements across the internet. The person sends the messages on the bulletin boards frequently to the victim. He can enter into the chat-rooms and distribed the victim by constantly sending emails.
			\item Cyber bullying:
		\end{enumerate}
	
	\subsection*{7: Enumerate the constitution and jurisdictions of Cyber Tribunal and Cyber Appellate Tribunal.}
	\subsubsection*{Cyber Tribunal}
	Book: page 384.
	
	\subsubsection{Cyber Appellate Tribunal}
	Book: page 388.
		
	\end{justify}
	
\end{document}