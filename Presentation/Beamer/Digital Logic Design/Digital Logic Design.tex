\documentclass{beamer}

\usepackage[utf8]{inputenc}
\usepackage
{
 authblk,
}
\usepackage[fleqn, tbtags]{mathtools}


\usetheme{AnnArbor}
\usecolortheme{beaver}
\usefonttheme{default}
\useinnertheme{rounded}
%\useoutertheme{}
\setbeamercolor{normal text}{bg=gray!3}


\title[CSE203]{Computer Logic Design Course}
\subtitle{Digital Logic Design}
\author[S. I. Kiron]
{
	Sofiullah Iqbal Kiron \\
	\href{mailto:sofiul.k.1023@gmail.com}{sofiul.k.1023@gmail.com}
}
\date[8 Jan]{January 8, 2021}
\institute[BSMRSTU]{Bongobondhu Sheikh Mujibur Rahman Science and Technology University}
\affil{Department of CSE}
\logo{\includegraphics[scale=0.05]{../Counting Principles/Logo.png}}


%It will show currentsection by using feature of \tableofcontents at the beginning of each section.
\AtBeginSection[]
{
 \begin{frame}
  \frametitle{Now we on\dots}
  \tableofcontents[currentsection]
 \end{frame}
}

\begin{document}
\frame{\titlepage}

\begin{frame}
\frametitle{Fulfilment}
\tableofcontents
\end{frame}

\section{Complements}

% Frame One Start.
\begin{frame}
\frametitle{Kind of Complements}
% \framesubtitle{}

There are two basic types of complements:
\begin{enumerate}
	\item $r$'s complement
	\item $(r-1)$'s complement
\end{enumerate}
\end{frame}
% Frame One Over.

% Frame Two Start.
\begin{frame}
\frametitle{$r$'s complement}

If there is a number called $N$ in base $r$ and that has $n$ number of digits in the integer part, then $r$'s complement of $N$ will be:
\[
\begin{rcases*}
r^n-N
\end{rcases*} \quad N\neq0
\]

\begin{block}{Sample}
$10$'s complement of $(52520)_10$ is $10^5-52520=47480$
\end{block}
\end{frame}
% Frame Two Over.

\end{document}