% Created at 19 January, 2022.

% Change the size of math notation inside math environment: https://tex.stackexchange.com/questions/348108/change-the-size-of-a-math-symbol
% longtable in beamer: \documentclass[table, compress]{beamer}

% Page transitions-> \transwipe[duration=0.6], \transdissolve, \transblindshorizontal, \transblindsvertical, \transboxin, \transcover, \transfade[duration=0.6], \transglitter, \transreplace and many many more with specificatinos and options.

% Beamer class don't support xcolor package options. How to load xcolor package options to the beamer class?


\documentclass[table, compress]{beamer}


% packages
\usepackage
{
	array, % control table-rule properties
	authblk, % Redefines \author command. Permits footnote type affiliation.
	awesomebox,
	caption, % I used it to remove buit-in caption tag from any figure by like \caption*{•}
	gensymb, % degree symbol
	graphicx,
	hyperref, % clickable hyper link
	listings,
	longtable, % another version of table that can extends through many pages. To use it in beamer class: \documentclass[table, compress]{beamer}
	multimedia, % sound, audio, video. Supported by beamer distro. Use media9 for another LaTeX class.
	qrcode,
	tabularx,
	transparent, % Control image/text transparency. OR, make a transparent photo from photoshop.
	xcolor,
}


% newcommand, renewcommand, macro
\newcommand{\myName}{Sofiullah Iqbal Kiron}
\newcommand{\myMail}{sofiul.k.1023@gmail.com}
\newcommand{\varsityName}{Bangobandhu Sheikh Mujibur Rahman Science and Technology University}
\newcommand{\varsityShortName}{BSMRSTU}
\newcommand{\dateOfSubmit}{24 January, 2022}


% titlepage
\title[Bitwise Operator]{Java Operators}
\subtitle{Bitwise Operators: $\&,\textrm{ }|,\textrm{ }>>,\textrm{ and }<<<$}
\author[S. I. Kiron]
{
	\myName \\
	\href{mailto:\myMail}{\myMail}
}
\date[Monday]{\dateOfSubmit}
\institute[\varsityShortName]{{\Large \textbf{\varsityName}}}
\affil{Department of CSE}
\logo{\includegraphics[scale=0.1]{../Counting Principles/Logo.png}}


% themes and appearance
\usetheme{Warsaw}
\usecolortheme{seahorse}
\usefonttheme{professionalfonts}
\useinnertheme{rounded}
\useoutertheme[footline=authorinstitutetitle]{miniframes}
\setbeamercolor{normal text}{bg=white, fg=black}
\setbeamercolor{frametitle}{fg=red!100, bg=green!20}
%\setbeamerfont{frametitle}{series=\bfseries}


% set-up issues %

% Rule set-up for table.
\setlength{\arrayrulewidth}{1pt}

% hyper/clickable link set-up.
\hypersetup
{
    colorlinks=true,
    linkcolor=blue,
    filecolor=magenta,      
    urlcolor=blue,
}

% Code listing set-up.
\lstset
{
	language = Java,
	backgroundcolor = \color{gray!10},
	basicstyle = \footnotesize\ttfamily, % Add all style properties without comma separator.
	keywordstyle = \color{blue},
	commentstyle = \color{green!90},
	showstringspaces = false,
	stringstyle = \color{red},
	captionpos = b %, b: bottom
 % numbers=left/right
}


% It will show currentsection by using feature of \tableofcontents at the beginning pf each section.
\AtBeginSection[]
{
 \begin{frame}
 
  \frametitle{Heads off to\dots}
  \tableofcontents[currentsection]
  
 \end{frame}
}


\begin{document}

{ % Local background surrendered in a pair of curly brace. Woanna add globally? put this command at the preamble.
\usebackgroundtemplate{\transparent{0.3}{\includegraphics[height=\paperheight,width=\paperwidth]{Pictures/background main.jpg}}}
	\frame
	{
		\titlepage
	}
}

% SECTION DEMO


% SECTION 1
\section{Preamble}


%%%%%%%%%%%%%%%%%%%%%%%%%%%%%%%%%%%%%%%%%%%%%%%%%%%%%%%%%%%%%%%%%%%%%%%%
\begin{frame}
    \frametitle{Start-UP}
    \begin{center}
    {\scriptsize
        Name: Md. Kazi Iqbal Hossen\\
        ID: 18ICTCSE065\\
        Department of CSE, BSMRSTU\\
    }
        \textcolor{green}{\rule{11 cm}{3 pt}} \\
        
        {\tiny
        \awesomebox[yellow][][]{2pt}{\faExclamationCircle}{red}{This presentation is created and animated using \LaTeX{}-Beamer class. Please keep it in full-screen view mode to take a better experience.}
        }
    \end{center}
    \texttt{\footnotesize Wanna explore beamer source code? click to:}
    {\footnotesize
    \begin{itemize}
        \item[Link 1:] \href{https://pastebin.ubuntu.com/p/dY8TWXY24C/}{\textit{\textbf{Ubuntu pastebin}}}
        \item[Link 2:] \href{https://github.com/Sofiullah-Iqbal-Kiron/LaTeX/blob/master/Presentation/Beamer/MAT205\%20Mid/MAT205\%20Mid.tex}{\textit{\textbf{Github}}}{\tiny (recommended)}
    \end{itemize}
    }
    Or, scan{\footnotesize(clickable in pdf)} the following quick response code:
    \begin{figure}
        \qrcode[hyperlink, padding, height=7mm, level=H, version=8]{https://github.com/Sofiullah-Iqbal-Kiron/LaTeX/blob/master/Presentation/Beamer/MAT205\%20Mid/MAT205\%20Mid.tex}
        {\small \caption*{QR}} % usepackage caption
        \centering
    \end{figure}
    \transsplitverticalout[duration=0.8]
\end{frame}
%%%%%%%%%%%%%%%%%%%%%%%%%%%%%%%%%%%%%%%%%%%%%%%%%%%%%%%%%%%%%%%%%%%%%%%%


%%%%%%%%%%%%%%%%%%%%%%%%%%%%%%%%%%%%%%%%%%%%%%%%%%%%%%%%%%%%%%%%%%%%%%%%
\begin{frame}
    \frametitle{Basic Beyond}
    \begin{block}{Info}
    	Java store values in memory as a binary string, except char type.
    \end{block}
    
    That's why Java provides bitwise operators to operate on them when we are regardless at focusing on original data.
   
\end{frame}
%%%%%%%%%%%%%%%%%%%%%%%%%%%%%%%%%%%%%%%%%%%%%%%%%%%%%%%%%%%%%%%%%%%%%%%%


%%%%%%%%%%%%%%%%%%%%%%%%%%%%%%%%%%%%%%%%%%%%%%%%%%%%%%%%%%%%%%%%%%%%%%%%
\begin{frame}
    \begin{figure}[!h]
    \centering
    	\includegraphics[scale=2.3]{E:/Pictures/WallPaper/Java WallPaper/istockphoto-518002738-170667a.jpg}
    \caption{Java, the Programming Language}
    \end{figure}
    
   
\end{frame}
%%%%%%%%%%%%%%%%%%%%%%%%%%%%%%%%%%%%%%%%%%%%%%%%%%%%%%%%%%%%%%%%%%%%%%%%


% SECTION 2
\section{Deep Dive}


%%%%%%%%%%%%%%%%%%%%%%%%%%%%%%%%%%%%%%%%%%%%%%%%%%%%%%%%%%%%%%%%%%%%%%%%
\begin{frame}
    \frametitle{Java's Bitwise Operator}
    \framesubtitle{Tabular Illustration}
    
\begin{center}
\rowcolors{3}{green!30}{green!60}
\arrayrulecolor{white}
\columnseprule = 0.2 mm
	\begin{longtable}{|| m{5 em} || m{10 em} ||}
		\hline\hline
		\rowcolor{teal!20}
		\multicolumn{2}{c}{\textbf{\textsf{\textcolor{black}{Bitwise Operators}}}}\\
		\hline\hline
		 Operators & Operation\\
		 $\thicksim$ & Unary NOT\\
		 \& & AND\\
 		 \textbf{\textbar} & OR\\
		 $\wedge$ & XOR\\
		 $>>$ & Right Shift\\
		 $>>>$ & Unsigned Right Shift\\
		 $<<$ & Left Shift\\
		\hline\hline
	\end{longtable}
\end{center}

\end{frame}
%%%%%%%%%%%%%%%%%%%%%%%%%%%%%%%%%%%%%%%%%%%%%%%%%%%%%%%%%%%%%%%%%%%%%%%%


%%%%%%%%%%%%%%%%%%%%%%%%%%%%%%%%%%%%%%%%%%%%%%%%%%%%%%%%%%%%%%%%%%%%%%%%
\begin{frame}
    \frametitle{Unary NOT}
    \begin{block}{$\thicksim$}
    	Also known as bitwise complement.
    \end{block}
    Inverts all the bits that is, if $0$ then $1$ otherwise $0$.
    \arrayrulecolor{black}
    \begin{table}
    \centering
    \begin{tabular}{| c  c |}
    	\hline
    	A & {\footnotesize $\bar{\textrm{A}}$}\\ \hline\hline
    	0 & 1\\ \hline
    	1 & 0\\ \hline
    \end{tabular}
    \caption*{\textbf{Truth Table}}
    \end{table}

\end{frame}
%%%%%%%%%%%%%%%%%%%%%%%%%%%%%%%%%%%%%%%%%%%%%%%%%%%%%%%%%%%%%%%%%%%%%%%%


%%%%%%%%%%%%%%%%%%%%%%%%%%%%%%%%%%%%%%%%%%%%%%%%%%%%%%%%%%%%%%%%%%%%%%%%
\begin{frame}
    \frametitle{Unary NOT}
    \framesubtitle{An example}
    \begin{example}
   	\begin{itemize}
   		\item[{\footnotesize $A\rightarrow$}] $1000\textrm{ }1111$
   		\item[{\footnotesize $\bar{A}\rightarrow$}] $0111\textrm{ }0000$
 	\end{itemize}
    \end{example}

\end{frame}
%%%%%%%%%%%%%%%%%%%%%%%%%%%%%%%%%%%%%%%%%%%%%%%%%%%%%%%%%%%%%%%%%%%%%%%%


%%%%%%%%%%%%%%%%%%%%%%%%%%%%%%%%%%%%%%%%%%%%%%%%%%%%%%%%%%%%%%%%%%%%%%%%
\begin{frame}
    \frametitle{Bitwise AND}
    
    \begin{block}{\&}
    	Operates bool logical AND operation on every bit of the given numbers.
    \end{block}
    
    \begin{table}
    \centering
    \begin{tabular}{| c  c c |}
    	\hline
    	A & B & A \& B\\ \hline\hline
    	0 & 0 & 0 \\ \hline
    	0 & 1 & 0 \\ \hline
    	1 & 0 & 0 \\ \hline
    	1 & 1 & 1 \\ \hline
    \end{tabular}
    \caption*{\textbf{Truth Table}}
    \end{table}
\end{frame}
%%%%%%%%%%%%%%%%%%%%%%%%%%%%%%%%%%%%%%%%%%%%%%%%%%%%%%%%%%%%%%%%%%%%%%%%


%%%%%%%%%%%%%%%%%%%%%%%%%%%%%%%%%%%%%%%%%%%%%%%%%%%%%%%%%%%%%%%%%%%%%%%%
\begin{frame}
    \frametitle{Bitwise AND}
    \framesubtitle{An example}
    
    \begin{table}
    \centering
    \begin{tabular}{c  c}
    	$42\rightarrow$ & $0010\textrm{ }1010$\\
    	$15\rightarrow$ & $0000\textrm{ }1111$\\ \hline
    	$42\textrm{ \& }15 = 10\rightarrow$ & $0000\textrm{ }1010$\\
    \end{tabular}
    \end{table}
\end{frame}
%%%%%%%%%%%%%%%%%%%%%%%%%%%%%%%%%%%%%%%%%%%%%%%%%%%%%%%%%%%%%%%%%%%%%%%%


%%%%%%%%%%%%%%%%%%%%%%%%%%%%%%%%%%%%%%%%%%%%%%%%%%%%%%%%%%%%%%%%%%%%%%%%
\begin{frame}
    \frametitle{Bitwise OR}
    \begin{block}{Bitwise OR: \textbf{\textbar}}
    	Combine bits such that if at least one bit is 1 then the resultant bit become 1.
    \end{block}
    
    \begin{table}
    \centering
    \begin{tabular}{| c  c c |}
    	\hline
    	A & B & A \textbf{\textbar} B\\ \hline\hline
    	0 & 0 & 0 \\ \hline
    	0 & 1 & 1 \\ \hline
    	1 & 0 & 1 \\ \hline
    	1 & 1 & 1 \\ \hline
    \end{tabular}
    \caption*{\textbf{Truth Table}}
    \end{table}

\end{frame}
%%%%%%%%%%%%%%%%%%%%%%%%%%%%%%%%%%%%%%%%%%%%%%%%%%%%%%%%%%%%%%%%%%%%%%%%


%%%%%%%%%%%%%%%%%%%%%%%%%%%%%%%%%%%%%%%%%%%%%%%%%%%%%%%%%%%%%%%%%%%%%%%%
\begin{frame}
    \frametitle{Bitwise OR}
    \framesubtitle{An example}
    
    \begin{table}
    \centering
    \begin{tabular}{c  c}
    	$42\rightarrow$ & $0010\textrm{ }1010$\\
    	$15\rightarrow$ & $0000\textrm{ }1111$\\ \hline
    	$42$ \textbf{\textbar} $15 = 47\rightarrow$ & $0010\textrm{ }1111$\\
    \end{tabular}
    \end{table}
\end{frame}
%%%%%%%%%%%%%%%%%%%%%%%%%%%%%%%%%%%%%%%%%%%%%%%%%%%%%%%%%%%%%%%%%%%%%%%%


%%%%%%%%%%%%%%%%%%%%%%%%%%%%%%%%%%%%%%%%%%%%%%%%%%%%%%%%%%%%%%%%%%%%%%%%
\begin{frame}
    \frametitle{Bitwise XOR}
    
    \begin{block}{Bitwise XOR: $\wedge$}
    	Combines bits such that when exactly one bit is 1 then the result is 1 otherwise 0.
    \end{block}
    
    \begin{itemize}
    	\item There is a useful property for programmers, if the second bit is 1 then the first bit is inverted.
    	\item Or, if the second bit is 0 then the first bit remains same.
    \end{itemize}
    
    \begin{table}
    \centering
    \begin{tabular}{| c  c c |}
    	\hline
    	A & B & A $\wedge$ B\\ \hline\hline
    	0 & 0 & 0 \\ \hline
    	0 & 1 & 1 \\ \hline
    	1 & 0 & 1 \\ \hline
    	1 & 1 & 0 \\ \hline
    \end{tabular}
    \caption*{\textbf{Truth Table}}
    \end{table}

\end{frame}
%%%%%%%%%%%%%%%%%%%%%%%%%%%%%%%%%%%%%%%%%%%%%%%%%%%%%%%%%%%%%%%%%%%%%%%%


%%%%%%%%%%%%%%%%%%%%%%%%%%%%%%%%%%%%%%%%%%%%%%%%%%%%%%%%%%%%%%%%%%%%%%%%
\begin{frame}
    \frametitle{Bitwise XOR}
    \framesubtitle{An example}
    
    \begin{table}
    \centering
    \begin{tabular}{c  c}
    	$42\rightarrow$ & $0010\textrm{ }1010$\\
    	$15\rightarrow$ & $0000\textrm{ }1111$\\ \hline
    	$42$ $\wedge$ $15 = 37\rightarrow$ & $0010\textrm{ }0101$\\
    \end{tabular}
    \end{table}
\end{frame}
%%%%%%%%%%%%%%%%%%%%%%%%%%%%%%%%%%%%%%%%%%%%%%%%%%%%%%%%%%%%%%%%%%%%%%%%


%%%%%%%%%%%%%%%%%%%%%%%%%%%%%%%%%%%%%%%%%%%%%%%%%%%%%%%%%%%%%%%%%%%%%%%%
\begin{frame}
    \frametitle{Right Shift}
    \begin{block}{Right Shift: $>>$}
    	Shifts all the bits in a value to the right a specified number of times.
    \end{block}
    First of all the value will be promoted to be int and then shifted by the specified number of times.\\
    Right shift is an efficient way for programmers to dividing the number by $2$ or multiplying it by $2^{-n}$.

\end{frame}
%%%%%%%%%%%%%%%%%%%%%%%%%%%%%%%%%%%%%%%%%%%%%%%%%%%%%%%%%%%%%%%%%%%%%%%%


%%%%%%%%%%%%%%%%%%%%%%%%%%%%%%%%%%%%%%%%%%%%%%%%%%%%%%%%%%%%%%%%%%%%%%%%
\begin{frame}
    \frametitle{Right Shift}
    \framesubtitle{An example}
%    \begin{lstlisting}
%    class RightShift {
%    
%    }
%    \end{lstlisting}
%    output in a tcolorbox

\end{frame}
%%%%%%%%%%%%%%%%%%%%%%%%%%%%%%%%%%%%%%%%%%%%%%%%%%%%%%%%%%%%%%%%%%%%%%%%


%%%%%%%%%%%%%%%%%%%%%%%%%%%%%%%%%%%%%%%%%%%%%%%%%%%%%%%%%%%%%%%%%%%%%%%%
\begin{frame}
    \frametitle{Left Shift}

\end{frame}
%%%%%%%%%%%%%%%%%%%%%%%%%%%%%%%%%%%%%%%%%%%%%%%%%%%%%%%%%%%%%%%%%%%%%%%%


%%%%%%%%%%%%%%%%%%%%%%%%%%%%%%%%%%%%%%%%%%%%%%%%%%%%%%%%%%%%%%%%%%%%%%%%
\begin{frame}
    \frametitle{Left Shift}
    \framesubtitle{An example}

\end{frame}
%%%%%%%%%%%%%%%%%%%%%%%%%%%%%%%%%%%%%%%%%%%%%%%%%%%%%%%%%%%%%%%%%%%%%%%%


\section{In the end}


\end{document}
