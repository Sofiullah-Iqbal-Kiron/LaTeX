\documentclass{beamer}

\usepackage[utf8]{inputenc}
\usepackage
{
 authblk,
}


\usetheme{AnnArbor}
\usecolortheme{beaver}
\usefonttheme{default}
\useinnertheme{rounded}
%\useoutertheme{}
\setbeamercolor{normal text}{bg=red!5}


\title[Counting]{Basic Counting Principle}
\subtitle{Combinatorics}
\author[Kiron]{Sofiullah Iqbal Kiron \\ \texttt{sofiul.k.1023@gmail.com}}
\date[Pohela Boishakh]{14 April, 2020}
\institute[BSMRSTU]{Bongobondhu Sheikh Mujibur Rahman Science and Technology University}
\affil{Department of CSE}
\logo{\includegraphics[height=1 cm]{Logo.png}}


%It will show currentsection by using feature of \tableofcontents at the beginning pf each section.
\AtBeginSection[]
{
 \begin{frame}
  \frametitle{Now we on\dots}
  \tableofcontents[currentsection]
 \end{frame}
}


\begin{document}

\frame{\titlepage}

\begin{frame}
\frametitle{Fulfilment}
\tableofcontents
\end{frame}

\section{5.1}
%Frame 1,
\begin{frame}
\frametitle{Two Basic Counting Principle}

\begin{enumerate}
\item Product Rule\pause
\item Sum Rule\pause
\end{enumerate}

\begin{columns}
\column{0.4\textwidth}
\textbf{Product Rule:} Suppose that a procedure can be broken down into a sequence of two tasks. If there are $n_1$ ways to do the first task and for each of these ways of doing the first task, there are $n_2$ ways to do the second task, then there are $n_1n_2$ ways to do the procedure.\pause
\transwipe

\column{0.5\textwidth}
\textbf{Sum Rule:} If a task can be done either in one of $n_1$ ways or in one of $n_2$ ways, where none of the set of $n_1$ ways is same as the any of set $n_2$ ways. Then there are $n_1+n_2$ ways to do the task.
\end{columns}
\end{frame}
%Frame 1 over.

%Frame 2,
\begin{frame}
\frametitle{Inclusion-Exclusion Principle}
Suppose that a task can be done in $n_1$ ways or in $n_2$ ways, but that some of the set of $n_1$ ways is same as some of the set of $n_2$ ways.
\\
To \alert{correctly count} the ways to do the two task:
\\We add\\
\begin{itemize}
\item<1-> The number of ways to do it in one way
\item<2-> The number of ways to do it in another way
\end{itemize}
then subtract.
\\
\begin{block}{So\dots}
The number of ways to do the task in a way that is both among the set of $n_1$ ways and the set of $n_2$ ways.\\
This technique is called \textbf{Inclusion-Exclusion}.
\end{block}
\end{frame}
%Frame 2 over.

\section{5.2}

%Frame 3,
\begin{frame}
\frametitle{The pigeonhole principle}
Let's consider that, there are 10 pigeon and 9 pigeonhole at your home. So, there must a hole that contains two pigeon.\pause\\
\begin{center}
\begin{tabular}{|c | c | c|}
\hline
\includegraphics[width=30 px]{Pigeon.png} & \includegraphics[width=30 px]{Pigeon.png} & \includegraphics[width=30 px]{Pigeon.png}\\
\hline
\includegraphics[width=30 px]{Pigeon.png}\includegraphics[width=30 px]{Pigeon.png} & \includegraphics[width=30 px]{Pigeon.png} & \includegraphics[width=30 px]{Pigeon.png}\\
\hline
\includegraphics[width=30 px]{Pigeon.png} & \includegraphics[width=30 px]{Pigeon.png} & \includegraphics[width=30 px]{Pigeon.png}\\
\hline
\end{tabular}\pause
\end{center}

\begin{block}{Principle:}
$k$ is a positive integer. There are $n$ objects $(n>k)$ placed in $k$ boxes, then there are at least one box containing two or more objects.
\end{block}
\end{frame}
%Frame 3 over.

%Frame 4,
\begin{frame}
\begin{block}{Corollary 1}
A function $(f)$ from set with $k+1$ of more elements to a set of $k$ elements, not $one-to-one$.
\end{block}
\textbf{Proof:} We can proof this by pigeonhole principle. Suppose elements of $x$ is pigeon(Domain), $y$ elements are pigeonhole(Co-domain). Then there are at least one pigeonhole(Co-domain) that contains more than one element.\\
That's mean, the function is not $one-to-one$.
\end{frame}
%Frame 4 over.

%Frame 5,
\begin{frame}
\frametitle{Generalized Pigeonhole Principle}
If $N$ objects are placed into $K$ boxes, then there is at least one box that containing at least $ceil(N/K)$ elements.
\end{frame}
%Frame 5 over.

%Frame 6,
\begin{frame}
\frametitle{Generalized Pigeonhole Principle}
If $N$ objects are placed into $K$ boxes, then there is at least one box that containing at least $ceil(N/K)$ \textcolor{red}{elements}.
\end{frame}
%Frame 6 over.

\end{document}