% ctrl + space: switch between editor and embedded viewer.

\documentclass[11 pt]{article}

\usepackage[document]{ragged2e}
\usepackage[left=3cm, right=2cm, top=2cm, bottom=2cm]{geometry}

\title{What I have learned During Photoshop Course}
\author{Sofiullah Iqbal Kiron}

\newcommand{\I}{\item}

\begin{document}

\maketitle
\tableofcontents


\justify
\begin{enumerate}
	\item alt + tab: for preview all currently opened apps in the middle of screen.
	\item If no files opened, photoshop home window will be views with recent photo related documents.
	\item Photoshop document extension is "PSD". Means - Photoshop document.
	\item Create new PSD by clicking "Create New" at the Home window.
	\item In the main working space, there are, Menubar on top most level of the window.
	\item Photo preview area at the left middle.
	\item Vertical aligned toolbar at left most side.
	\item Options bar for currently selected tool just below of Menubar.
	\item So many control panels on right side.
	\item Preferred four photoshop document type: phg, JPG, tif and PSD.
	\item Can change color of preview canvas background color by right click and selection.
	\item Changing color theme preview icons by pressing: ctrl + alt + shift + click
	\item There is a useful color picker on panels bar of any standards like, RGB, HSB and so forth\dots
	\item Rich tool tip are enabled. Can remove this from Edit $\rightarrow$ Preferences $\rightarrow$ Tools
	\item Can customize toolbar from Edit $\rightarrow$ Toolbar
	\item Object selection tool (W) is more useful. So pretty.
	\item Must be able to working with layers and masks.
	\item Photoshop's ability to work with layers is definitely one of favorite features. Let's dive into to see how we can work with multiple images in a single document. Layers will be shown in panel section of photoshop interface. Or we can hide/show this layer panel from window $\rightarrow$ Layers or directly pressing  F7. Now, we can drag layers from one file and then drag this to another.
	\item ctrl + r: In order to show rulers. We can change rulers unit(pixel, percent etc) by right click above on ruler and then drag ruler guides from rulers.
	\item Photoshop's most powerful image creation tool, layer. Every photo's begins life is flat, no-layers image file. In photoshop, this called background. We can float one image upon another and those all are several. Edit on one image don't affect to other. But we can blend layers together. A document that contains layers is said to be layered composition. Welcome to the power of layers. Layers are stored as stack. It is an example of stack. "alt" button often reverses the natural behavior of an icon/action.
	\item[Canvas:] The physical parameter of the image.
	\item[Windows:] enter key.
	\item[Mac:] return key.
	\item[Magic Wand Tool:] Select same colored pixels.
	\item Multiply blending mode will remove bright pixels. It just keep the dark staff.
	\item[HSB:] Hue, Saturation, Brightness.
	\item[Brush Essentials:] 'b' stands for brush tool. Brush means brush and we can draw by brush on a selected layer. Set the brush property form option bar. To reset brush, just right click and press reset. We can change brush size, hardness, shape and angle also. Can search brush or select a brush preset. Click on folder brush icon in order to bring up brush property panel that hold brush presets, size, shape, hardness and so forth(A number of additional options). Press or hold right square bracket to increase brush size or left square bracket to decrease the size. Another way to increase/decrease brush size, alt + mouse right button press and hold and then right-left "to and fro", up-down "to and fro" for hardness. Press and hold shift key to order the brush paint only straight line, no matter where did you drag your brush. Check all options from options bar for the brush tool. Or we can change opacity of brush color by directly pressing 0-9 keys. 0 to 100\%, 1 to 10\% etc.
	\item Crtl + Backspace to fill the layer with background color. Flow will increase/decrease of painting flow/speed.
	\item Photo Retouching is a art to making something look better. Ctrl + Alt + J to copy currently selected layer and make a new layer, also bring up the rename dialogue box. Can use arrow keys to move up/down/left/right the keys.
	\item 'X' to call swap(foreground\_color, background\_color)
	\item Ctrl + h to hide/show the selection outline.
	\item[Spot Healing Brush:] Allows us to paint with content aware fill. Also we can invoke fill by pressing Shift + Backspace(windows)/delete(mac). Shortcut of healing brush tool: J. It's one kind of surgery, g $\rightarrow$ j. Pretty good.
	\item ctrl + tab: preview next image.
	\item shift + F1/F2 to go to darker/brighter interface. There are two darker and two brighter interface. Preview pasteboard.
	\item ctrl + 0: Fit to page.
	\item ctrl + , (Comma): Hide/Show current layer.
	\item ctrl + ; (Semicolon): Hide/Show guides if present.
	\item ctrl + shift + .(>) will increase the selected font size 1 point at per click.
	\item ctrl + shift + ,(<) will decrease the selected font size 1 point at per click.
	\item ctrl + alt + shift + ,(>) will increase the selected font size 5 point at per click.
	\item ctrl + alt + shift + ,(<) will decrease the selected font size 5 point at per click.
	\item shift + click\_on\_thumbnail will hide layer mask. A red cross sign will appear on the layer thumbnail.
\end{enumerate}

\section{File Format}
Basically there are three file format: Capture format, Working format and the Output format. JPEG is the default file format that uses by mobile phone usually. The Output format: PSD(Max 2GB) and TIFF(Max 4GB) both saves all the information and use lossless compression. Other shareable formats, JPEG, PNG, GIF, MOV, MP3, Photoshop and so on\dots JPEG is more simple to upload and download. GIF is the animated photo.

% Setting counter of section manually.
\setcounter{section}{10}
\section{Content Aware}
This is the intermediate section. We can design magazine poster in photoshop. Photoshop uses artificial intelligence for content aware. Let's use patch(Tali Dewa/JoraTali Dewa) tool. Patch tool replaces a selected area by taking pixels from another part of the same picture. It is more useful than "Healing Brush Tool" (Healing Brush Tool fill brushed area by content-aware fill). The patch tool is useful for making big modification very quickly. Select patch tool, shortcut: j. Then select the object section as you select by a lasso tool. Or we can select by any other object selection tool then patch it. Or press and hold alt to create a polygonal selection outline. \\
Content-Aware Patch:

\section{Creating and Formatting Text}
This is a section of working with text. Photoshop provides a type tool: T, that can crate layers of text.

\section{Layer Effect/Style(fx)}
A layer style is a combination of blends and layer effects. Double click on fx written icon on the right side of layer or click on fx button below of layer panel in order to bring up the layer style option box. Using global light will crate a shadow with the full frame. Pressing "alt" key is always amazing. "alt" when layer style option box is opened, "Cancel" button will changed to "Reset". Pressing "alt" when you are in local interface, will active "character shortcuts" of top menu bar. In "Drop Shadow", here is a unique option, "Layer Knocks Out Drop Shadow". If this option is unchecked then layer will no be visible and just left of the drop shadow.

\section{Smart Object}
Photoshop never let lose the quality of a smart object. If we convert a vector graphics or image to a smart object it will never lose its original quality even if we rescale or resmaple it or change it's position. Methods to bring a vector graphics or image and convert it to smart object in photoshop:
\begin{enumerate}
	\I Mark and copy a vector object from illustrator (ctrl + c) and paste on photoshop workspace (ctrl + v). 
\end{enumerate}
Smart Objects never lose quality, cheers!

\end{document}
