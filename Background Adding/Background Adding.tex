\documentclass[10pt, a4paper]{article}

\usepackage[utf8]{inputenc}
\usepackage
{
	amsmath,
	amsfonts,
	amssymb,
	authblk, % Author affilation.
	background, % For adding background image.
	fontawesome,
	mVersion,
	hyperref,
	pdfpages,
	watermark
}
\usepackage[left=2cm, right=2cm, top=2cm, bottom=3cm]{geometry}
\usepackage[document]{ragged2e} % Various text alignment.
\usepackage[short, nodayofweek, level, 12hr]{datetime}

% To change background image options later on:
% Can use \backgroundsetup{<options>} at anywhere at the document. Not necessarily to add it on the preamble.
% Universal.
\backgroundsetup
{
	%contents={\includegraphics[width=300px]{Java vs Python, deep thinking..png}}
	%contents={\includegraphics[width=80px, height=90px]{../../../Downloads/Delete This after.jpg}}, % Text, Image.
	color=green!99,
	scale=10, % Can put any positive value but 1-10 is good enough.
	placement=center, % top, center, bottom: relative to page, not text.
	angle=1,
	opacity=0
}

\setVersion{0.0}
\increaseBuild % Will update version at each recompilation.

\title{Background Package Practice}
\author
{
	Sofiullah Iqbal Kiron\\
	\href{mailto:sofiul.k.1023@gmail.com}{sofiul.k.1023@gmail.com}
}
\date{18 January, 2021 \\ \currenttime \\ Version: \version}
\affil
{
	BSMRSTU, Department of CSE \\
	SHIICT \\
	{\tiny Copyright\faCopyright\hspace{2pt} under Sofiullah Book Agency Publishing Section}
}


\begin{document}

\includepdf{F:/Torrent/TorrentBD/Adobe Photoshop/2. Photoshop CC One-on-One Intermediate (2 of 4)/Exercise Files/15_shapes/CMYK cover.pdf}

\maketitle

\justify
\thispageheading{A Heading Occurs Here}
This a practice for adding background image. For further details, see manual.pdf or download it from CTAN's official website. The package offers the placement of background material on the pages of a document. The user can control many aspects(contents, position, color, opacity) of the background material that will be displayed; all placement and attribute settings are controlled by setting key values. \\ \\
%After all text copied from "Assignment of Moinul vai", Importance section. \textit{\textbf{Copied Starts:}} A country can move forward if it has a good source of Power Generation with the Transmission and Distribution system. By the amount of power generation, we can determine the country’s overall development. We will find lots of countries those who ensure a well-established power system and they are leading the world. It also can depend on many sources from where a country will generate electric power. Our energy needs (electricity needs) can be met from various of sources, like so - hydel, thermal (includes fossil fuel and nuclear fuel plants), wind power, solar (PV) and so on\dots. Out of the above, wind power and solar are not reliable for continuous production of electricity - wind can die down, solar won’t work in night or when the cloud cover is too thick. Hydel plants are seasonal - they depend on rivers, dams, rains etc. Thermal power plants are the most reliable - they can work round the clock, round the year (except for occasional downtime for maintenance). It is not therefore surprising that many countries use thermal power as the base to meet the energy demand and use the other modes of energy production to supplement the thermal units. For example, some of the hydel units may be operated in the night when solar units are not available and similarly when wind power units are running, solar units may be used to charge large battery banks so that the power can be drawn when the wind units are down. Thus power grid management uses a mix of base load and peak load energy sources - thermal power plants being used for base load management. A country can move forward if it has a good source of Power Generation with the Transmission and Distribution system. By the amount of power generation, we can determine the country’s overall development. We will find lots of countries those who ensure a well-established power system and they are leading the world. It also can depend on many sources from where a country will generate electric power. Our energy needs (electricity needs) can be met from various of sources, like so - hydel, thermal (includes fossil fuel and nuclear fuel plants), wind power, solar (PV) and so on\dots. Out of the above, wind power and solar are not reliable for continuous production of electricity - wind can die down, solar won’t work in night or when the cloud cover is too thick. \\ \\
%\newgeometry{margin=5cm}
%Hydel plants are seasonal - they depend on rivers, dams, rains etc. Thermal power plants are the most reliable - they can work round the clock, round the year (except for occasional downtime for maintenance). It is not therefore surprising that many countries use thermal power as the base to meet the energy demand and use the other modes of energy production to supplement the thermal units. For example, some of the hydel units may be operated in the night when solar units are not available and similarly when wind power units are running, solar units may be used to charge large battery banks so that the power can be drawn when the wind units are down. Thus power grid management uses a mix of base load and peak load energy sources - thermal power plants being used for base load management. A country can move forward if it has a good source of Power Generation with the Transmission and Distribution system. By the amount of power generation, we can determine the country’s overall development. We will find lots of countries those who ensure a well-established power system and they are leading the world. It also can depend on many sources from where a country will generate electric power. Our energy needs (electricity needs) can be met from various of sources, like so - hydel, thermal (includes fossil fuel and nuclear fuel plants), wind power, solar (PV) and so on\dots. Out of the above, wind power and solar are not reliable for continuous production of electricity - wind can die down, solar won’t work in night or when the cloud cover is too thick. Hydel plants are seasonal - they depend on rivers, dams, rains etc. Thermal power plants are the most reliable - they can work round the clock, round the year (except for occasional downtime for maintenance). It is not therefore surprising that many countries use thermal power as the base to meet the energy demand and use the other modes of energy production to supplement the thermal units. For example, some of the hydel units may be operated in the night when solar units are not available and similarly when wind power units are running, solar units may be used to charge large battery banks so that the power can be drawn when the wind units are down. Thus power grid management uses a mix of base load and peak load energy sources - thermal power plants being used for base load management. A country can move forward if it has a good source of Power Generation with the Transmission and Distribution system. By the amount of power generation, we can determine the country’s overall development. We will find lots of countries those who ensure a well-established power system and they are leading the world. It also can depend on many sources from where a country will generate electric power. Our energy needs (electricity needs) can be met from various of sources, like so - hydel, thermal (includes fossil fuel and nuclear fuel plants), wind power, solar (PV) and so on\dots. Out of the above, wind power and solar are not reliable for continuous production of electricity - wind can die down, solar won’t work in night or when the cloud cover is too thick. Hydel plants are seasonal - they depend on rivers, dams, rains etc. Thermal power plants are the most reliable - they can work round the clock, round the year (except for occasional downtime for maintenance). It is not therefore surprising that many countries use thermal power as the base to meet the energy demand and use the other modes of energy production to supplement the thermal units. For example, some of the hydel units may be operated in the night when solar units are not available and similarly when wind power units are running, solar units may be used to charge large battery banks so that the power can be drawn when the wind units are down. Thus power grid management uses a mix of base load and peak load energy sources - thermal power plants being used for base load management. A country can move forward if it has a good source of Power Generation with the Transmission and Distribution system. By the amount of power generation, we can determine the country’s overall development. We will find lots of countries those who ensure a well-established power system and they are leading the world. It also can depend on many sources from where a country will generate electric power. Our energy needs (electricity needs) can be met from various of sources, like so - hydel, thermal (includes fossil fuel and nuclear fuel plants), wind power, solar (PV) and so on\dots. Out of the above, wind power and solar are not reliable for continuous production of electricity - wind can die down, solar won’t work in night or when the cloud cover is too thick. Hydel plants are seasonal - they depend on rivers, dams, rains etc. Thermal power plants are the most reliable - they can work round the clock, round the year (except for occasional downtime for maintenance). It is not therefore surprising that many countries use thermal power as the base to meet the energy demand and use the other modes of energy production to supplement the thermal units. For example, some of the hydel units may be operated in the night when solar units are not available and similarly when wind power units are running, solar units may be used to charge large battery banks so that the power can be drawn when the wind units are down. Thus power grid management uses a mix of base load and peak load energy sources - thermal power plants being used for base load management. A country can move forward if it has a good source of Power Generation with the Transmission and Distribution system. By the amount of power generation, we can determine the country’s overall development. We will find lots of countries those who ensure a well-established power system and they are leading the world. It also can depend on many sources from where a country will generate electric power. Our energy needs (electricity needs) can be met from various of sources, like so - hydel, thermal (includes fossil fuel and nuclear fuel plants), wind power, solar (PV) and so on\dots. Out of the above, wind power and solar are not reliable for continuous production of electricity - wind can die down, solar won’t work in night or when the cloud cover is too thick. Hydel plants are seasonal - they depend on rivers, dams, rains etc. Thermal power plants are the most reliable - they can work round the clock, round the year (except for occasional downtime for maintenance). It is not therefore surprising that many countries use thermal power as the base to meet the energy demand and use the other modes of energy production to supplement the thermal units. For example, some of the hydel units may be operated in the night when solar units are not available and similarly when wind power units are running, solar units may be used to charge large battery banks so that the power can be drawn when the wind units are down. Thus power grid management uses a mix of base load and peak load energy sources - thermal power plants being used for base load management. A country can move forward if it has a good source of Power Generation with the Transmission and Distribution system. By the amount of power generation, we can determine the country’s overall development. We will find lots of countries those who ensure a well-established power system and they are leading the world. It also can depend on many sources from where a country will generate electric power. Our energy needs (electricity needs) can be met from various of sources, like so - hydel, thermal (includes fossil fuel and nuclear fuel plants), wind power, solar (PV) and so on\dots. Out of the above, wind power and solar are not reliable for continuous production of electricity - wind can die down, solar won’t work in night or when the cloud cover is too thick. Hydel plants are seasonal - they depend on rivers, dams, rains etc. Thermal power plants are the most reliable - they can work round the clock, round the year (except for occasional downtime for maintenance). It is not therefore surprising that many countries use thermal power as the base to meet the energy demand and use the other modes of energy production to supplement the thermal units. For example, some of the hydel units may be operated in the night when solar units are not available and similarly when wind power units are running, solar units may be used to charge large battery banks so that the power can be drawn when the wind units are down. Thus power grid management uses a mix of base load and peak load energy sources - thermal power plants being used for base load management. A country can move forward if it has a good source of Power Generation with the Transmission and Distribution system. By the amount of power generation, we can determine the country’s overall development. We will find lots of countries those who ensure a well-established power system and they are leading the world. It also can depend on many sources from where a country will generate electric power. Our energy needs (electricity needs) can be met from various of sources, like so - hydel, thermal (includes fossil fuel and nuclear fuel plants), wind power, solar (PV) and so on\dots. Out of the above, wind power and solar are not reliable for continuous production of electricity - wind can die down, solar won’t work in night or when the cloud cover is too thick. Hydel plants are seasonal - they depend on rivers, dams, rains etc. Thermal power plants are the most reliable - they can work round the clock, round the year (except for occasional downtime for maintenance). It is not therefore surprising that many countries use thermal power as the base to meet the energy demand and use the other modes of energy production to supplement the thermal units. For example, some of the hydel units may be operated in the night when solar units are not available and similarly when wind power units are running, solar units may be used to charge large battery banks so that the power can be drawn when the wind units are down. Thus power grid management uses a mix of base load and peak load energy sources - thermal power plants being used for base load management. A country can move forward if it has a good source of Power Generation with the Transmission and Distribution system. By the amount of power generation, we can determine the country’s overall development. We will find lots of countries those who ensure a well-established power system and they are leading the world. It also can depend on many sources from where a country will generate electric power. Our energy needs (electricity needs) can be met from various of sources, like so - hydel, thermal (includes fossil fuel and nuclear fuel plants), wind power, solar (PV) and so on\dots. Out of the above, wind power and solar are not reliable for continuous production of electricity - wind can die down, solar won’t work in night or when the cloud cover is too thick. Hydel plants are seasonal - they depend on rivers, dams, rains etc. Thermal power plants are the most reliable - they can work round the clock, round the year (except for occasional downtime for maintenance). It is not therefore surprising that many countries use thermal power as the base to meet the energy demand and use the other modes of energy production to supplement the thermal units. For example, some of the hydel units may be operated in the night when solar units are not available and similarly when wind power units are running, solar units may be used to charge large battery banks so that the power can be drawn when the wind units are down. Thus power grid management uses a mix of base load and peak load energy sources - thermal power plants being used for base load management. A country can move forward if it has a good source of Power Generation with the Transmission and Distribution system. By the amount of power generation, we can determine the country’s overall development. We will find lots of countries those who ensure a well-established power system and they are leading the world. It also can depend on many sources from where a country will generate electric power. Our energy needs (electricity needs) can be met from various of sources, like so - hydel, thermal (includes fossil fuel and nuclear fuel plants), wind power, solar (PV) and so on\dots. Out of the above, wind power and solar are not reliable for continuous production of electricity - wind can die down, solar won’t work in night or when the cloud cover is too thick. Hydel plants are seasonal - they depend on rivers, dams, rains etc. Thermal power plants are the most reliable - they can work round the clock, round the year (except for occasional downtime for maintenance). It is not therefore surprising that many countries use thermal power as the base to meet the energy demand and use the other modes of energy production to supplement the thermal units. For example, some of the hydel units may be operated in the night when solar units are not available and similarly when wind power units are running, solar units may be used to charge large battery banks so that the power can be drawn when the wind units are down. Thus power grid management uses a mix of base load and peak load energy sources - thermal power plants being used for base load management. A country can move forward if it has a good source of Power Generation with the Transmission and Distribution system. By the amount of power generation, we can determine the country’s overall development. We will find lots of countries those who ensure a well-established power system and they are leading the world. It also can depend on many sources from where a country will generate electric power. Our energy needs (electricity needs) can be met from various of sources, like so - hydel, thermal (includes fossil fuel and nuclear fuel plants), wind power, solar (PV) and so on\dots. Out of the above, wind power and solar are not reliable for continuous production of electricity - wind can die down, solar won’t work in night or when the cloud cover is too thick. Hydel plants are seasonal - they depend on rivers, dams, rains etc. Thermal power plants are the most reliable - they can work round the clock, round the year (except for occasional downtime for maintenance). It is not therefore surprising that many countries use thermal power as the base to meet the energy demand and use the other modes of energy production to supplement the thermal units. For example, some of the hydel units may be operated in the night when solar units are not available and similarly when wind power units are running, solar units may be used to charge large battery banks so that the power can be drawn when the wind units are down. Thus power grid management uses a mix of base load and peak load energy sources - thermal power plants being used for base load management. A country can move forward if it has a good source of Power Generation with the Transmission and Distribution system. By the amount of power generation, we can determine the country’s overall development. We will find lots of countries those who ensure a well-established power system and they are leading the world. It also can depend on many sources from where a country will generate electric power. Our energy needs (electricity needs) can be met from various of sources, like so - hydel, thermal (includes fossil fuel and nuclear fuel plants), wind power, solar (PV) and so on\dots. Out of the above, wind power and solar are not reliable for continuous production of electricity - wind can die down, solar won’t work in night or when the cloud cover is too thick. Hydel plants are seasonal - they depend on rivers, dams, rains etc. Thermal power plants are the most reliable - they can work round the clock, round the year (except for occasional downtime for maintenance). It is not therefore surprising that many countries use thermal power as the base to meet the energy demand and use the other modes of energy production to supplement the thermal units. For example, some of the hydel units may be operated in the night when solar units are not available and similarly when wind power units are running, solar units may be used to charge large battery banks so that the power can be drawn when the wind units are down. Thus power grid management uses a mix of base load and peak load energy sources - thermal power plants being used for base load management. A country can move forward if it has a good source of Power Generation with the Transmission and Distribution system. By the amount of power generation, we can determine the country’s overall development. We will find lots of countries those who ensure a well-established power system and they are leading the world. It also can depend on many sources from where a country will generate electric power. Our energy needs (electricity needs) can be met from various of sources, like so - hydel, thermal (includes fossil fuel and nuclear fuel plants), wind power, solar (PV) and so on\dots. Out of the above, wind power and solar are not reliable for continuous production of electricity - wind can die down, solar won’t work in night or when the cloud cover is too thick. Hydel plants are seasonal - they depend on rivers, dams, rains etc. Thermal power plants are the most reliable - they can work round the clock, round the year (except for occasional downtime for maintenance). It is not therefore surprising that many countries use thermal power as the base to meet the energy demand and use the other modes of energy production to supplement the thermal units. For example, some of the hydel units may be operated in the night when solar units are not available and similarly when wind power units are running, solar units may be used to charge large battery banks so that the power can be drawn when the wind units are down. Thus power grid management uses a mix of base load and peak load energy sources - thermal power plants being used for base load management. A country can move forward if it has a good source of Power Generation with the Transmission and Distribution system. By the amount of power generation, we can determine the country’s overall development. We will find lots of countries those who ensure a well-established power system and they are leading the world. It also can depend on many sources from where a country will generate electric power. Our energy needs (electricity needs) can be met from various of sources, like so - hydel, thermal (includes fossil fuel and nuclear fuel plants), wind power, solar (PV) and so on\dots. Out of the above, wind power and solar are not reliable for continuous production of electricity - wind can die down, solar won’t work in night or when the cloud cover is too thick. Hydel plants are seasonal - they depend on rivers, dams, rains etc. Thermal power plants are the most reliable - they can work round the clock, round the year (except for occasional downtime for maintenance). It is not therefore surprising that many countries use thermal power as the base to meet the energy demand and use the other modes of energy production to supplement the thermal units. For example, some of the hydel units may be operated in the night when solar units are not available and similarly when wind power units are running, solar units may be used to charge large battery banks so that the power can be drawn when the wind units are down. Thus power grid management uses a mix of base load and peak load energy sources - thermal power plants being used for base load management. A country can move forward if it has a good source of Power Generation with the Transmission and Distribution system. By the amount of power generation, we can determine the country’s overall development. We will find lots of countries those who ensure a well-established power system and they are leading the world. It also can depend on many sources from where a country will generate electric power. Our energy needs (electricity needs) can be met from various of sources, like so - hydel, thermal (includes fossil fuel and nuclear fuel plants), wind power, solar (PV) and so on\dots. Out of the above, wind power and solar are not reliable for continuous production of electricity - wind can die down, solar won’t work in night or when the cloud cover is too thick. Hydel plants are seasonal - they depend on rivers, dams, rains etc. Thermal power plants are the most reliable - they can work round the clock, round the year (except for occasional downtime for maintenance). It is not therefore surprising that many countries use thermal power as the base to meet the energy demand and use the other modes of energy production to supplement the thermal units. For example, some of the hydel units may be operated in the night when solar units are not available and similarly when wind power units are running, solar units may be used to charge large battery banks so that the power can be drawn when the wind units are down. Thus power grid management uses a mix of base load and peak load energy sources - thermal power plants being used for base load management. A country can move forward if it has a good source of Power Generation with the Transmission and Distribution system. By the amount of power generation, we can determine the country’s overall development. We will find lots of countries those who ensure a well-established power system and they are leading the world. It also can depend on many sources from where a country will generate electric power. Our energy needs (electricity needs) can be met from various of sources, like so - hydel, thermal (includes fossil fuel and nuclear fuel plants), wind power, solar (PV) and so on\dots. Out of the above, wind power and solar are not reliable for continuous production of electricity - wind can die down, solar won’t work in night or when the cloud cover is too thick. Hydel plants are seasonal - they depend on rivers, dams, rains etc. Thermal power plants are the most reliable - they can work round the clock, round the year (except for occasional downtime for maintenance). It is not therefore surprising that many countries use thermal power as the base to meet the energy demand and use the other modes of energy production to supplement the thermal units. For example, some of the hydel units may be operated in the night when solar units are not available and similarly when wind power units are running, solar units may be used to charge large battery banks so that the power can be drawn when the wind units are down. Thus power grid management uses a mix of base load and peak load energy sources - thermal power plants being used for base load management. A country can move forward if it has a good source of Power Generation with the Transmission and Distribution system. By the amount of power generation, we can determine the country’s overall development. We will find lots of countries those who ensure a well-established power system and they are leading the world. It also can depend on many sources from where a country will generate electric power. Our energy needs (electricity needs) can be met from various of sources, like so - hydel, thermal (includes fossil fuel and nuclear fuel plants), wind power, solar (PV) and so on\dots. Out of the above, wind power and solar are not reliable for continuous production of electricity - wind can die down, solar won’t work in night or when the cloud cover is too thick. Hydel plants are seasonal - they depend on rivers, dams, rains etc. Thermal power plants are the most reliable - they can work round the clock, round the year (except for occasional downtime for maintenance). It is not therefore surprising that many countries use thermal power as the base to meet the energy demand and use the other modes of energy production to supplement the thermal units. For example, some of the hydel units may be operated in the night when solar units are not available and similarly when wind power units are running, solar units may be used to charge large battery banks so that the power can be drawn when the wind units are down. Thus power grid management uses a mix of base load and peak load energy sources - thermal power plants being used for base load management. A country can move forward if it has a good source of Power Generation with the Transmission and Distribution system. By the amount of power generation, we can determine the country’s overall development. We will find lots of countries those who ensure a well-established power system and they are leading the world. It also can depend on many sources from where a country will generate electric power. Our energy needs (electricity needs) can be met from various of sources, like so - hydel, thermal (includes fossil fuel and nuclear fuel plants), wind power, solar (PV) and so on\dots. Out of the above, wind power and solar are not reliable for continuous production of electricity - wind can die down, solar won’t work in night or when the cloud cover is too thick. Hydel plants are seasonal - they depend on rivers, dams, rains etc. Thermal power plants are the most reliable - they can work round the clock, round the year (except for occasional downtime for maintenance). It is not therefore surprising that many countries use thermal power as the base to meet the energy demand and use the other modes of energy production to supplement the thermal units. For example, some of the hydel units may be operated in the night when solar units are not available and similarly when wind power units are running, solar units may be used to charge large battery banks so that the power can be drawn when the wind units are down. Thus power grid management uses a mix of base load and peak load energy sources - thermal power plants being used for base load management. A country can move forward if it has a good source of Power Generation with the Transmission and Distribution system. By the amount of power generation, we can determine the country’s overall development. We will find lots of countries those who ensure a well-established power system and they are leading the world. It also can depend on many sources from where a country will generate electric power. Our energy needs (electricity needs) can be met from various of sources, like so - hydel, thermal (includes fossil fuel and nuclear fuel plants), wind power, solar (PV) and so on\dots. Out of the above, wind power and solar are not reliable for continuous production of electricity - wind can die down, solar won’t work in night or when the cloud cover is too thick. Hydel plants are seasonal - they depend on rivers, dams, rains etc. Thermal power plants are the most reliable - they can work round the clock, round the year (except for occasional downtime for maintenance). It is not therefore surprising that many countries use thermal power as the base to meet the energy demand and use the other modes of energy production to supplement the thermal units. For example, some of the hydel units may be operated in the night when solar units are not available and similarly when wind power units are running, solar units may be used to charge large battery banks so that the power can be drawn when the wind units are down. Thus power grid management uses a mix of base load and peak load energy sources - thermal power plants being used for base load management. A country can move forward if it has a good source of Power Generation with the Transmission and Distribution system. By the amount of power generation, we can determine the country’s overall development. We will find lots of countries those who ensure a well-established power system and they are leading the world. It also can depend on many sources from where a country will generate electric power. Our energy needs (electricity needs) can be met from various of sources, like so - hydel, thermal (includes fossil fuel and nuclear fuel plants), wind power, solar (PV) and so on\dots. Out of the above, wind power and solar are not reliable for continuous production of electricity - wind can die down, solar won’t work in night or when the cloud cover is too thick. Hydel plants are seasonal - they depend on rivers, dams, rains etc. Thermal power plants are the most reliable - they can work round the clock, round the year (except for occasional downtime for maintenance). It is not therefore surprising that many countries use thermal power as the base to meet the energy demand and use the other modes of energy production to supplement the thermal units. For example, some of the hydel units may be operated in the night when solar units are not available and similarly when wind power units are running, solar units may be used to charge large battery banks so that the power can be drawn when the wind units are down. Thus power grid management uses a mix of base load and peak load energy sources - thermal power plants being used for base load management. A country can move forward if it has a good source of Power Generation with the Transmission and Distribution system. By the amount of power generation, we can determine the country’s overall development. We will find lots of countries those who ensure a well-established power system and they are leading the world. It also can depend on many sources from where a country will generate electric power. Our energy needs (electricity needs) can be met from various of sources, like so - hydel, thermal (includes fossil fuel and nuclear fuel plants), wind power, solar (PV) and so on\dots. Out of the above, wind power and solar are not reliable for continuous production of electricity - wind can die down, solar won’t work in night or when the cloud cover is too thick. Hydel plants are seasonal - they depend on rivers, dams, rains etc. Thermal power plants are the most reliable - they can work round the clock, round the year (except for occasional downtime for maintenance). It is not therefore surprising that many countries use thermal power as the base to meet the energy demand and use the other modes of energy production to supplement the thermal units. For example, some of the hydel units may be operated in the night when solar units are not available and similarly when wind power units are running, solar units may be used to charge large battery banks so that the power can be drawn when the wind units are down. Thus power grid management uses a mix of base load and peak load energy sources - thermal power plants being used for base load management. A country can move forward if it has a good source of Power Generation with the Transmission and Distribution system. By the amount of power generation, we can determine the country’s overall development. We will find lots of countries those who ensure a well-established power system and they are leading the world. It also can depend on many sources from where a country will generate electric power. Our energy needs (electricity needs) can be met from various of sources, like so - hydel, thermal (includes fossil fuel and nuclear fuel plants), wind power, solar (PV) and so on\dots. Out of the above, wind power and solar are not reliable for continuous production of electricity - wind can die down, solar won’t work in night or when the cloud cover is too thick. Hydel plants are seasonal - they depend on rivers, dams, rains etc. Thermal power plants are the most reliable - they can work round the clock, round the year (except for occasional downtime for maintenance). It is not therefore surprising that many countries use thermal power as the base to meet the energy demand and use the other modes of energy production to supplement the thermal units. For example, some of the hydel units may be operated in the night when solar units are not available and similarly when wind power units are running, solar units may be used to charge large battery banks so that the power can be drawn when the wind units are down. Thus power grid management uses a mix of base load and peak load energy sources - thermal power plants being used for base load management. A country can move forward if it has a good source of Power Generation with the Transmission and Distribution system. By the amount of power generation, we can determine the country’s overall development. We will find lots of countries those who ensure a well-established power system and they are leading the world. It also can depend on many sources from where a country will generate electric power. Our energy needs (electricity needs) can be met from various of sources, like so - hydel, thermal (includes fossil fuel and nuclear fuel plants), wind power, solar (PV) and so on\dots. Out of the above, wind power and solar are not reliable for continuous production of electricity - wind can die down, solar won’t work in night or when the cloud cover is too thick. Hydel plants are seasonal - they depend on rivers, dams, rains etc. Thermal power plants are the most reliable - they can work round the clock, round the year (except for occasional downtime for maintenance). It is not therefore surprising that many countries use thermal power as the base to meet the energy demand and use the other modes of energy production to supplement the thermal units. For example, some of the hydel units may be operated in the night when solar units are not available and similarly when wind power units are running, solar units may be used to charge large battery banks so that the power can be drawn when the wind units are down. Thus power grid management uses a mix of base load and peak load energy sources - thermal power plants being used for base load management. A country can move forward if it has a good source of Power Generation with the Transmission and Distribution system. By the amount of power generation, we can determine the country’s overall development. We will find lots of countries those who ensure a well-established power system and they are leading the world. It also can depend on many sources from where a country will generate electric power. Our energy needs (electricity needs) can be met from various of sources, like so - hydel, thermal (includes fossil fuel and nuclear fuel plants), wind power, solar (PV) and so on\dots. Out of the above, wind power and solar are not reliable for continuous production of electricity - wind can die down, solar won’t work in night or when the cloud cover is too thick. Hydel plants are seasonal - they depend on rivers, dams, rains etc. Thermal power plants are the most reliable - they can work round the clock, round the year (except for occasional downtime for maintenance). It is not therefore surprising that many countries use thermal power as the base to meet the energy demand and use the other modes of energy production to supplement the thermal units. For example, some of the hydel units may be operated in the night when solar units are not available and similarly when wind power units are running, solar units may be used to charge large battery banks so that the power can be drawn when the wind units are down. Thus power grid management uses a mix of base load and peak load energy sources - thermal power plants being used for base load management. \textit{\textbf{Copied Over.}}

\end{document}